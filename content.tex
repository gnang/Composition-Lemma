`% In this file you should put the actual content of the blueprint.
% It will be used both by the web and the print version.
% It should *not* include the \begin{document}
%
% If you want to split the blueprint content into several files then
% the current file can be a simple sequence of \input. Otherwise It
% can start with a \section or \chapter for instance.

\chapter{\texorpdfstring{Composition Lemma}{Composition Lemma}}

\section{Overview}
The \textit{Composition Lemma} was developed and refined over 6 years, beginning in 2018, as a novel approach to settle in the affirmative the \textit{Graceful Tree Conjecture}. The first of such papers was posted in \cite{gnang2020gracefullabelingstrees} by Gnang. A further developed series of papers resolving the same conjecture again appeared in \cite{gnang2022compositionlemma} and \cite{gnang2023proofkotzigringelrosaconjecture}. Recently, the same method has been applied to settle other longstanding conjectures in \cite{chalise2024treenedgesdecomposes} and \cite{chalise2024prooftreepackingconjecture}. We comment that the series of papers shared on the open-source platform arXiv reflect the evolving landscape of Gnang's thought process, and the frequent re-uploads were driven by the natural progression and refinement of ideas. However, we recognize that these numerous edits may have unintentionally caused confusion and raised questions regarding the success of the method. In the current work, we aim to address these concerns by presenting a detailed blueprint of the proof, with the goal of formalizing it in Lean4.

\section{Functional Directed Graphs}\label{sec:Functional Directed Graphs}
For notational convenience, let $\mathbb{Z}_{n}$ denote the set whose members are the first $n$ natural numbers, i.e.,
\begin{equation}
\mathbb{Z}_{n}:=\big\{0,\ldots,\,n-1\big\}.
\end{equation}
For a function $f:\mathbb{Z}_{m}\to\mathbb{Z}_{n}$, we write $f\in\mathbb{Z}_{n}^{\mathbb{Z}_{m}}$.
For $X\subseteq\mathbb{Z}_{m}$, $f(X)$ denotes the image of $X$
under $f$, i.e.,
\begin{equation}
f(X)=\{f(i):i\in X\},
\end{equation}
and $|f(X)|$ denotes its
cardinality. For $Y\subseteq\mathbb{Z}_{n}\ensuremath{,}f^{-1}(Y)$
denotes the pre-image of $Y$ under $f$ i.e.
\begin{equation}
f^{-1}(Y)=\{j\in\mathbb{Z}_{m}:f(j)\in Y\}
\end{equation}
\begin{definition}[Functional digraphs]\label{defn:functional-directed-graphs}
% REVIEW THE LEAN LABELS BELLOW:
  %\lean{}
  %\leanok
  %\uses{defn:standard-simplex}
For an arbitrary $f\in \mathbb{Z}_n^{\mathbb{Z}_n}$, the \emph{functional directed graph} prescribed by $f$, denoted $G_f$, is such that the vertex set $V(G_f)$ and the directed edge set $E(G_f)$ are respectively as follows:
\[
V(G_f) = \mathbb{Z}_n, \; E(G_f) = \{(v,f(v)):v \in \mathbb{Z}_n\}.
\]
\end{definition}
\begin{definition}[Graceful functional digraphs]\label{defn:graceful-functional-graphs}
  %\lean{}
  %\leanok
  %\uses{defn:functional-directed-graphs}
The functional directed graph prescribed by $f\in\mathbb{Z}_{n}^{\mathbb{Z}_{n}}$ is graceful if there exist
a bijection $\sigma\in \text{S}_n \subset
 \mathbb{Z}_{n}^{\mathbb{Z}_{n}}$ such that
\begin{equation}
\big\{\left|\sigma f\sigma^{-1}(i)-i\right|:i\in\mathbb{Z}_{n}\big\}=\mathbb{Z}_{n}.
\end{equation}
If $\sigma=\text{id}$ (the identity function), then $G_{f}$ --- the functional directed graph prescribed by $f$ --- is gracefully labeled. 
\end{definition}
\begin{defn}[Automorphism group]\label{defn:aut-functional-graphs}
  %\lean{}
  %\leanok
  %\uses{defn:functional-directed-graphs}
For a functional directed graph $G_f$, its automorphism group, denoted $\text{Aut}\left(G_f\right)$, is defined as follows:
\[
\text{Aut}\left(G_f\right) = \large\{\sigma \in \text{S}_n : \{(i, f(i)): i \in \mathbb{Z}_n \} = \{(j, \sigma f \sigma^{-1} (j)) : j \in \mathbb{Z}_n\}\large\}. 
\]
For a polynomial $P \in \mathbb{C}[x_0, \ldots, x_{n-1}]$, its automorphism group, denoted $\text{Aut($P$)}$, is defined as follows:
\[
\text{Aut}\left(P\right)=\large\{\sigma\in\text{S}_{n}:P\left(x_{0},\ldots,x_{i},\ldots,x_{n-1}\right)=P\left(x_{\sigma(0)},\ldots,x_{\sigma(i)},\ldots,x_{\sigma(n-1)}\right)\large\}.
\]
\end{defn}
\begin{definition}[Graceful re-labelings]\label{defn:graceful-functional-graphs-set}
  %\lean{}
  %\leanok
  %\uses{defn:functional-directed-graphs, defn:graceful-functional-graphs}
The set of distinct gracefully labeled functional directed graphs isomorphic to $G_{f}$ is 
\[
\text{GrL}\left(G_{f}\right)\,:=\left\{ G_{\sigma f\sigma^{-1}}:\begin{array}{c}
\sigma\text{ is a representative of a coset in }\nicefrac{\text{S}_{n}}{\text{Aut}\left(G_{f}\right)}\text{ and }\\
\mathbb{Z}_{n}=\left\{ \left|\sigma f\sigma^{-1}\left(i\right)-i\right|\,:\,i\in\mathbb{Z}_{n}\right\} 
\end{array}\right\} 
\]
\end{definition}
\begin{definition}[Complementary labeling involution]\label{defn:complementary-labeling-symmetry}
  %\lean{}
  %\leanok
  %\uses{defn:functional-directed-graphs, defn:graceful-functional-graphs}
If $\varphi=n-1-\text{id}$, i.e. $\varphi \in \mathbb{Z}_{n}^{\mathbb{Z}_{n}}$ such that
\[
\varphi(i)=n-1-i,\, \forall \,i\in \mathbb{Z}_{n},
\]
then for an arbitrary $f\in\mathbb{Z}_{n}^{\mathbb{Z}_{n}}$ the complementary labeling involution is defined as the map
\[
f \mapsto \varphi f \varphi^{-1}
\]
\end{definition}
Observe that for all $f\in \mathbb{Z}_{n}^{\mathbb{Z}_{n}}$ the complementary labeling involution fixes the induced edge label of each edge as seen from the equality
\begin{equation}
\left|f(i)-i\right|=\left|\varphi f(i)-\varphi(i)\right|,\quad\forall\,i\in\mathbb{Z}_{n}.
\end{equation}
In other words, induced edge labels are fixed by the vertex relabeling effected by $\varphi$. We call this induced edge label symmetry the \emph{complementary labeling symmetry} of the functional directed graph $G_f$.

\section{Quotient-Remainder Theorem and Lagrange Interpolation}\label{sec:QRT}
 \begin{proposition}[Multivariate Quotient-Remainder]\label{prop:multivariate-quotient-remainder}
  %\lean{Algebra.Polynomial.FieldDivision}
  %\leanok
  %\uses{}
Let $d(x)\in\mathbb{C}[x]$ be a degree $n$ monic polynomial with
simple roots, i.e.,
\begin{equation}
d(x)=\prod_{i\in\mathbb{Z}_{n}}(x-\alpha_{i})\;\text{ and }\:0\ne\prod_{0\le u<v<n}(\alpha_{v}-\alpha_{u}),
\end{equation}
where $\{\alpha_{u}:u\in\mathbb{Z}_{n}\}\subset\mathbb{C}$. 
For
all $P\in\mathbb{C}[x_{0},\ldots,x_{m-1}]$, there exists a unique remainder
$r(x_{0},\ldots,x_{m-1})\in\mathbb{C}[x_{0},\ldots,x_{m-1}]$ of degree
at most $n-1$ in each variable such that for quotients: $\big\{ q_{k}(x_{0},\ldots,x_{n-1}):k\in\mathbb{Z}_{n}\big\}\subset\mathbb{C}[x_{0},\ldots,x_{n-1}]$, we have
\begin{equation}
P(x_{0},\ldots,x_{m-1})= r(x_{0},\ldots,x_{m-1})+\sum_{u\in\mathbb{Z}_{m}}q_{u}(x_{0},\ldots,x_{m-1})\,d(x_{u}).
\end{equation}
 \end{proposition}
\begin{proof}
  %\leanok
We prove by induction on the number of variables that the remainder admits the expansion
\begin{equation}
r(x_{0},\ldots,x_{m-1})=\sum_{g\in\mathbb{Z}_{n}^{\mathbb{Z}_{m}}}P(\boldsymbol{\alpha}_{g})\prod_{i\in\mathbb{Z}_{m}}\left(\prod_{j_{i}\in\mathbb{Z}_{n}\backslash\{g(i)\}}\bigg(\frac{x_{i}-\alpha_{j_{i}}}{\alpha_{g(i)}-\alpha_{j_{i}}}\bigg)\right),
\end{equation}
where for notational convenience $P(\alpha_{g}):=P(\alpha_{g(0)},\ldots,\alpha_{g(m-1)})$.
The base case stems from the univariate quotient-remainder theorem over the field $\mathbb{C}$. The univariate-quotient remainder theorem over the field $\mathbb{C}$ asserts that there exist
a unique quotient-remainder pair $\big(q(x_{0}),\,r(x_{0})\big)\in\mathbb{C}[x_{0}]\times\mathbb{C}[x_{0}]$
subject to
\begin{equation}
H(x_{0})=q(x_{0})\,d(x_{0})+r(x_{0}),
\end{equation}
where $r(x_{0})\in\mathbb{C}[x_{0}]$ is of degree at most $n-1$.
It is completely determined by its evaluation over $\{\alpha_{i}:i\in\mathbb{Z}_{n}\}$,
and by Lagrange interpolation we have 
\begin{equation}
r(x_{0})=\sum_{g\in\mathbb{Z}_{n}^{\mathbb{Z}_{1}}}H(\alpha_{g(0)})\prod_{j_{0}\in\mathbb{Z}_{n}\backslash\{g(0)\}}\bigg(\frac{x_{0}-\alpha_{j_{0}}}{\alpha_{g(0)}-\alpha_{j_{0}}}\bigg),
\end{equation}
thus establishing the claim in the base case. For the induction step,
assume as our induction hypothesis that for all $F\in\mathbb{C}\left[x_{0},\ldots,x_{m-1}\right]$,
we have
\begin{equation}
F=\sum_{k\in\mathbb{Z}_{m}}q_{k}(x_{0},\ldots,x_{m-1})\,d(x_{k})+\sum_{g\in\mathbb{Z}_{n}^{\mathbb{Z}_{m}}}F(\alpha_{g})\prod_{i\in\mathbb{Z}_{m}}\left(\prod_{j_{i}\in\mathbb{Z}_{n}\backslash\left\{ g(i)\right\} }\left(\frac{x_{i}-\alpha_{j_{i}}}{\alpha_{g(i)}-\alpha_{j_{i}}}\right)\right).
\end{equation}
We proceed to show that the hypothesis implies that every polynomial in
$m+1$ variables also admits a similar expansion, thus establishing
the desired claim. Consider a polynomial $H \in \mathbb{C}[x_{0},\ldots,x_{m}]$. We view $H$ as a univariate polynomial in the variable $x_{m}$ whose coefficients lie in the field of fraction $\mathbb{C}(x_{0},\ldots,x_{m-1})$.
The univariate quotient-remainder theorem over the field of fractions $\mathbb{C}(x_{0},\ldots,x_{m-1})$ asserts that
there exit a unique quotient-remainder pair
\[
\big(q(x_{m}),\,r(x_{m})\big)\in\big(\mathbb{C}(x_{0},\ldots,x_{m-1})\big)[x_{m}]\times\big(\mathbb{C}(x_{0},\ldots,x_{m-1})\big)[x_{m}]
\]
subject to
\begin{equation}
H\big(x_{0},\ldots,x_{m}\big)=q(x_{0},\ldots,x_{m})\,d(x_{m})+r(x_{0},\ldots,x_{m}),
\end{equation}
where $r\left(x_{0},\ldots,x_{m}\right)\in\big(\mathbb{C}(x_{0},\ldots,x_{m-1})\big)[x_{m}]$
is of degree at most $n-1$ in the variable $x_{m}$. We write 
\begin{equation}
r\left(x_{0},\ldots,x_{m}\right)=\sum_{k\in\mathbb{Z}_{n}}a_{k}\left(x_{0},\ldots,x_{m-1}\right)\,(x_{m})^{k}.
\end{equation}
We now show that coefficients $\big\{ a_k(x_{0},\ldots,x_{m-1}):k\in \mathbb{Z}_n \big\}$ all lie in the polynomial ring $\mathbb{C}[x_{0},\ldots,x_{m-1}]$
via the equality 
\begin{equation}
\bigg(\text{Vander}\left(\begin{array}{c}
\alpha_{0}\\
\vdots\\
\alpha_{u}\\
\vdots\\
\alpha_{n-1}
\end{array}\right)\bigg)\cdot\left(\begin{array}{c}
a_{0}\left(x_{0},\ldots,x_{m-1}\right)\\
\vdots\\
a_{u}\left(x_{0},\ldots,x_{m-1}\right)\\
\vdots\\
a_{n-1}\left(x_{0},\ldots,x_{m-1}\right)
\end{array}\right)=\left(\begin{array}{c}
H(x_{0},\ldots,x_{m-1},\alpha_{0})\\
\vdots\\
H(x_{0},\ldots,x_{m-1},\alpha_{u})\\
\vdots\\
H(x_{0},\ldots,x_{m-1},\alpha_{n-1})
\end{array}\right),
\end{equation}
where 
\begin{equation}
\bigg(\text{Vander}\left(\begin{array}{c}
\alpha_{0}\\
\vdots\\
\alpha_{u}\\
\vdots\\
\alpha_{u}
\end{array}\right)\bigg)\left[i,j\right]=(\alpha_{i})^{j},\ \forall\,0\le i,j<n.
\end{equation}
Since the Vandermonde matrix is invertible by the fact
\begin{equation}
0\ne\det\bigg(\text{Vander}\left(\begin{array}{c}
\alpha_{0}\\
\vdots\\
\alpha_{u}\\
\vdots\\
\alpha_{u}
\end{array}\right)\bigg)=\prod_{0\le u<v<n}(\alpha_{v}-\alpha_{u}),
 \end{equation}
we indeed have 
\begin{equation}
\left(\begin{array}{c}
a_{0}\left(x_{0},\ldots,x_{m-1}\right)\\
\vdots\\
a_{u}\left(x_{0},\ldots,x_{m-1}\right)\\
\vdots\\
a_{n-1}\left(x_{0},\ldots,x_{m-1}\right)
\end{array}\right)=\bigg(\text{Vander}\left(\begin{array}{c}
\alpha_{0}\\
\vdots\\
\alpha_{u}\\
\vdots\\
\alpha_{u}
\end{array}\right)\bigg)^{-1}\cdot\left(\begin{array}{c}
H(x_{0},\ldots,x_{m-1},\alpha_{0})\\
\vdots\\
H(x_{0},\ldots,x_{m-1},\alpha_{u})\\
\vdots\\
H(x_{0},\ldots,x_{m-1},\alpha_{n-1})
\end{array}\right).
\end{equation}
Therefore, we have 
\begin{equation}
H\big(x_{0},\ldots,x_{m}\big)
= q_{m}\big(x_{0},\ldots,x_{m}\big)\,d(x_{m})+\sum_{g(m)\in\mathbb{Z}_{n}}H\big(x_{0},\ldots,x_{m-1},\alpha_{g(m)}\big)\prod_{j\in\mathbb{Z}_{n}\backslash\{g(m)\}}\left(\frac{x_{m}-\alpha_{j_{m}}}{\alpha_{g(m)}-\alpha_{j_{m}}}\right).
\end{equation}
Applying the induction hypothesis to coefficients
\[
\left\{ H\left(x_{0},\ldots,x_{m-1},\alpha_{g\left(m\right)}\right): \alpha_{g\left(m\right)}\in\mathbb{C}\right\} \subset\mathbb{C}[x_{0},\ldots,x_{m-1}]
\]
yields the desired expansion. Finally, quotients $\big\{ q_{k}(x_{0},\ldots,x_{m-1}):k\in\mathbb{Z}_{m}\big\}$ lie in the polynomial ring $\mathbb{C}[x_{0},\ldots,x_{m-1}]$ since the polynomial $H(x_0,\ldots,x_{m-1})-r(x_0,\ldots,x_{m-1})$ lies in the ideal generated by members of the set $\big\{ d(x_{u}):u\in\mathbb{Z}_{m}\big\}$.
\end{proof}
\begin{proposition}[Ring Homomorphism]\label{prop:ring-homomorphism}
  %\lean{Algebra.Polynomial.FieldDivision}
  %\leanok
  %\uses{}
For an arbitrary $H\in\mathbb{C}\left[x_{0},\ldots,x_{n-1}\right]$, let $\overline{H}$ denote
the {remainder} of the congruence class 
\[
H\text{ modulo the ideal generated by }\left\{ d(x_{i}):i\in\mathbb{Z}_{n}\right\} ,
\]
where
\[
d(x)=\prod_{i\in\mathbb{Z}_{n}}(x-\alpha_{i})\;\text{ and }\:0\ne\prod_{0\le u<v<n}(\alpha_{v}-\alpha_{u}),
\]
Then the following hold:
\begin{enumerate}
    \item For all $g\in\mathbb{Z}_{n}^{\mathbb{Z}_{n}}$, we have $\overline{H}(\boldsymbol{\alpha}_g)=H(\boldsymbol{\alpha}_g)$. 
    \item If $H = H_0 + H_1,$ where $H_0, H_1 \in \mathbb{C}\left[x_{0},\ldots,x_{n-1}\right]$, then $\overline{H_0} + \overline{H_1} = \overline{H}$.
    \item If $H = H_0 \cdot H_1,$ where $H_0, H_1 \in \mathbb{C}\left[x_{0},\ldots,x_{n-1}\right]$,  then $ \overline{H} \equiv \overline{H_0} \cdot \overline{H_1}$.
\end{enumerate}
 \end{proposition}
\begin{proof}
  %\leanok
  The first claim follows from Proposition \ref{prop:multivariate-quotient-remainder} for we see that the divisor vanishes over the lattice. To prove the second claim we recall that
  \[
\overline{H}=\sum_{g\in\mathbb{Z}_{n}^{\mathbb{Z}_{n}}}H(\boldsymbol{\alpha}_{g})\prod_{i\in\mathbb{Z}_{n}}\left(\prod_{j_{i}\in\mathbb{Z}_{n}\backslash\{g(i)\}}\bigg(\frac{x_{i}-\alpha_{j_{i}}}{\alpha_{g(i)}-\alpha_{j_{i}}}\bigg)\right),
  \]
  \[
  \implies\overline{H}=\sum_{g\in\mathbb{Z}_{n}^{\mathbb{Z}_{n}}}\big(H_{0}(\boldsymbol{\alpha}_{g})+H_{1}(\boldsymbol{\alpha}_{g})\big)\prod_{i\in\mathbb{Z}_{n}}\left(\prod_{j_{i}\in\mathbb{Z}_{n}\backslash\{g(i)\}}\bigg(\frac{x_{i}-\alpha_{j_{i}}}{\alpha_{g(i)}-\alpha_{j_{i}}}\bigg)\right),
  \]
  \[
  \implies\overline{H}=\sum_{k\in\mathbb{Z}_{2}}\sum_{g\in\mathbb{Z}_{n}^{\mathbb{Z}_{n}}}H_{k}(\boldsymbol{\alpha}_{g})\prod_{i\in\mathbb{Z}_{n}}\left(\prod_{j_{i}\in\mathbb{Z}_{n}\backslash\{g(i)\}}\bigg(\frac{x_{i}-\alpha_{j_{i}}}{\alpha_{g(i)}-\alpha_{j_{i}}}\bigg)\right).
  \]
  Thus $\overline{H_0} + \overline{H_1} = \overline{H}$ as claimed. Finally the fact (iii) is a straightforward consequence of Proposition \ref{prop:Orthogonality}, which is proved next.
\end{proof}
\begin{proposition}
 \label{prop:Orthogonality}
  %\lean{}
  %\leanok
  %\uses{prop:multivariate-quotient-remainder}   
 Let $f,g \in\mathbb{Z}_{n}^{\mathbb{Z}_{n}}$. For congruence classes prescribed modulo the ideal generated by $\{d(x_{i}):i \in \Z_n\}$, if
\[
d(x)=\prod_{i\in\mathbb{Z}_{n}}(x-\alpha_{i})\;\text{ such that }\:0\ne\prod_{0\le u<v<n}(\alpha_{v}-\alpha_{u}),
\]
then
\[
L_{f}({\bf x})\cdot L_{g}({\bf x})\equiv\begin{cases}
L_{f}({\bf x}) & \text{if }f=g\\
0 & \text{otherwise,}
\end{cases}
\]
\end{proposition}
\begin{proof} Observe that
\[
L_{f}(\mathbf{x})\cdot L_{g}(\mathbf{x})\;=\;\prod_{i\in\mathbb{Z}_{n}}\bigg(\big(c_{i,f}\frac{d(x_{i})}{x_{i}-\alpha_{f(i)}}\big)\big(c_{i,g}\frac{d(x_{i})}{x_{i}-\alpha_{g(i)}}\big)\bigg),
\]
where
\[
c_{i,f}=\prod_{j_{i}\in\mathbb{Z}_{n}\backslash\{f(i)\}}\big(\alpha_{f(i)}-\alpha_{j_{i}}\big)^{-1}\quad\text{ and }\quad c_{i,g}=\prod_{j_{i}\in\mathbb{Z}_{n}\backslash\{g(i)\}}\big(\alpha_{g(i)}-\alpha_{j_{i}}\big)^{-1}.
\]
If $f \neq g$, then there exists $j \in \Z_n$ such that $f(j)\neq g(j)$ and $L_{f}(\mathbf{x})\cdot L_{g}(\mathbf{x})$ is a multiple of $(x_{j})^{\underline{n}}$,
as a result of which we obtain $L_{f}(\mathbf{x})\cdot L_{g}(\mathbf{x}) \equiv 0$. Alternatively if $f = g$, then
\[
L_{f}(\mathbf{x})\cdot L_{g}(\mathbf{x})=\big(L_{f}(\mathbf{x})\big)^{2}=L_{f}(\mathbf{x})+\bigg(\big(L_{f}(\mathbf{x})\big)^{2}-L_{f}(\mathbf{x})\bigg).
\]
We now show that $\big(L_{f}(\mathbf{x})\big)^{2}-L_{f}(\mathbf{x})\equiv 0$ modulo the ideal generated by $\left\{ d(x_{i}):i\in\mathbb{Z}_{n}\right\}$.
\begin{align*}
\big(L_{f}(\mathbf{x})\big)^{2}-L_{f}(\mathbf{x})\; & =\;L_{f}(\mathbf{x})\left(L_{f}(\mathbf{x})-1\right)\\
 & =\;L_{f}(\mathbf{x})\bigg(L_{f}(\mathbf{x})-\sum_{g\in\mathbb{Z}_{n}^{\mathbb{Z}_{n}}}L_{g}\left({\bf x}\right)\bigg)\\
 & =\;-L_{f}(\mathbf{x})\bigg(\sum_{g\in\mathbb{Z}_{n}^{\mathbb{Z}_{n}}\setminus\{f\}}L_{g}\left({\bf x}\right)\bigg)\\
 & \equiv\;0,
\end{align*}
where the latter congruence identity stems from the prior setting where $f\ne g$.
\end{proof}

\begin{defn}[Polynomial of Grace]\label{defn:polynomial-grace-definition} We define $P_f \in \mathbb{C}[x_0, \ldots, x_{n-1}]$ for all $f \in \mathbb{Z}_n^{\mathbb{Z}_n}$ as follows:
\begin{equation}
    P_f(\mathbf{x}) \coloneq \underbrace{\prod_{0\le u<v<n}(x_{v}-x_{u})}_{V(x_0,\ldots,x_{n-1})}\,\underbrace{\prod_{0\le u<v<n}\big((x_{f(v)}-x_{v})^{2}-(x_{f(u)}-x_{u})^{2}\big)}_{E_f(x_0,\ldots,x_{n-1})}.
\end{equation}
\end{defn}
\begin{defn}[Congruence class]\label{defn:polynomial-congruence}
  %\lean{}  %\leanok
  %\uses{prop:multivariate-quotient-remainder}
For polynomials $P,Q \in \mathbb{C}[x_0, \ldots, x_{n-1}]$, if
\begin{equation}
P({\bf x}) \equiv \; Q({\bf x}) \; \mod\bigg\{\prod_{j\in\mathbb{Z}_{n}}(x_{i}-j):i\in\mathbb{Z}_{n}\bigg\},
\end{equation}
we simply write $P \equiv Q$.
\end{defn}
 Unless otherwise stated, all subsequent congruence identities are prescribed modulo the ideal of polynomials generated by members of the set
\[
\bigg\{\prod_{j\in\mathbb{Z}_{n}}(x_{i}-j):i\in\mathbb{Z}_{n}\bigg\}
\]
\begin{proposition}[Certificate of Grace]\label{prop:polynomial-grace-certificate}
  %\lean{}
  %\leanok
  %\uses{defn:polynomial-grace-definition, defn:polynomial-congruence, prop:multivariate-quotient-remainder}
Let $f\in\mathbb{Z}_{n}^{\mathbb{Z}_{n}}$. The functional
directed graph $G_f$ prescribed by $f$ is graceful if and only if $P_f(\mathbf{x}) \; \not\equiv \; 0$.
\end{proposition}
\begin{proof}
Observe that the vertex Vandermonde factor $V(\mathbf{x})$ is of degree exactly $n-1$ in each variable and therefore equal
to its remainder, i.e.,
\begin{equation}\label{eq:Vf}
V(\mathbf{x})=\sum_{\theta\in\text{S}_{n}}\text{sgn}(\theta)\prod_{i\in\mathbb{Z}_{n}}(x_{i})^{\theta(i)}=\prod_{v\in\mathbb{Z}_{n}}(v!)\;\sum_{\theta\in\text{S}_{n}}\text{sgn}(\theta)\prod_{\begin{array}{c}
\substack{i\in\mathbb{Z}_{n}\\
j_{i}\in\mathbb{Z}_{n}\backslash\{\theta(i)\}
}
\end{array}}\left(\frac{x_{i}-j_{i}}{\theta(i)-j_{i}}\right),
\end{equation}
where 
\begin{equation}
\text{sgn}(\theta):=\prod_{0\le u<v<n}\left(\frac{\theta(v)-\theta(u)}{v-u}\right),\quad\forall\,\theta\in\text{S}_{n}.
\end{equation}
When $n>2$, for every $f\in\mathbb{Z}_{n}^{\mathbb{Z}_{n}}$, the induced edge label Vandermonde factor $E_f(\mathbf{x})$
is of degree $>(n-1)$ in some of its variables. Therefore, by Proposition  \ref{prop:multivariate-quotient-remainder}, we have
\begin{equation}\label{eq:Ef}
E_{f}(\mathbf{x})=\sum_{l\in\mathbb{Z}_{m}}q_{l}(\mathbf{x})\prod_{k\in\mathbb{Z}_{n}}(x_{l}-k)+\prod_{v\in\mathbb{Z}_{n}}(v!)\frac{(n-1+v)!}{(2v)!}\sum_{\begin{array}{c}
g\in\mathbb{Z}_{n}^{\mathbb{Z}_{n}}\\
|gf-g|\in\text{S}_{n}
\end{array}}\text{sgn}(|gf-g|)\prod_{\begin{array}{c}
\substack{i\in\mathbb{Z}_{n}\\
j_{i}\in\mathbb{Z}_{n}\backslash\{g(i)\}
}
\end{array}}\big(\frac{x_{i}-j_{i}}{g(i)-j_{i}}\big).
\end{equation}
Observe that by the expansions in \ref{eq:Vf} and \ref{eq:Ef}, 
\[
P_{f}(\mathbf{x})=\sum_{l\in\mathbb{Z}_{m}}q_{l}(\mathbf{x})V(\mathbf{x})\prod_{k\in\mathbb{Z}_{n}}(x_{l}-k)+
\]
\[
\bigg(\prod_{v\in\mathbb{Z}_{n}}v!\sum_{\theta\in\text{S}_{n}}\text{sgn}(\theta)\prod_{\begin{array}{c}
\substack{i\in\mathbb{Z}_{n}\\
j_{i}\in\mathbb{Z}_{n}\backslash\{\theta(i)\}
}
\end{array}}\big(\frac{x_{i}-j_{i}}{\theta(i)-j_{i}}\big)\bigg)\bigg(\prod_{v\in\mathbb{Z}_{n}}(v!)\frac{(n-1+v)!}{(2v)!}\sum_{\begin{array}{c}
g\in\mathbb{Z}_{n}^{\mathbb{Z}_{n}}\\
|gf-g|\in\text{S}_{n}
\end{array}}\text{sgn}(|gf-g|)\prod_{\begin{array}{c}
\substack{i\in\mathbb{Z}_{n}\\
j_{i}\in\mathbb{Z}_{n}\backslash\{g(i)\}
}
\end{array}}\big(\frac{x_{i}-j_{i}}{g(i)-j_{i}}\big)\bigg).
\]
is congruent to
\begin{equation} \label{eq:graceful-evaluation}
\prod_{v\in\mathbb{Z}_{n}}(v!)^{2}\frac{(n-1+v)!}{(2v)!}\sum_{\begin{array}{c}
\substack{\sigma\in\text{S}_{n}\\
\text{s.t.}\\
\left|\sigma f-\sigma\right|\in\text{S}_{n}
}
\end{array}}\text{sgn}(\sigma\left|\sigma f-\sigma\right|)\prod_{\begin{array}{c}
\substack{i\in\mathbb{Z}_{n}\\
j_{i}\in\mathbb{Z}_{n}\backslash\{\sigma(i)\}
}
\end{array}}\big(\frac{x_{i}-j_{i}}{\sigma(i)-j_{i}}\big),
\end{equation}
where the permutation $|\sigma f-\sigma |$ denotes the induced edge label permutation associated with a graceful relabeling $G_{\sigma f\sigma^{-1}}$ of $G_f$.
The congruence above stems from Prop. \ref{prop:Orthogonality}. A graceful labeling necessitates the integer coefficient
\[
\prod_{0\le i<j<n}(j-i)\big(j^{2}-i^{2})=\prod_{0\le i<j<n}(j-i)^{2}(j+i)=\prod_{v\in\mathbb{Z}_{n}}\left(v!\right)^{2}\frac{\left(n-1+v\right)!}{\left(2v\right)!} \neq 0,
\]
thus establishing the desired claim.
\end{proof}
\begin{example} We present an example of a path on 5 vertices. This is known to be graceful, so we expect a non-zero remainder.
\ \\
\begin{center}   
\resizebox{!}{3em}{
\begin{tikzpicture}[>={Stealth[round]}, node distance=2cm and 1cm, every node/.style={circle, draw, minimum size=1cm}]
    % Define vertices
    \node (v0) at (0, 0) {0};
    \node (v1) at (3, 0) {1};
    \node (v2) at (6, 0) {2};
    \node (v3) at (9, 0) {3};
    \node (v4) at (12, 0) {4};
    
    % Draw edges
    \draw [->] (v1) to (v0); % Edge 1 -> 0
    \draw [->] (v2) to (v1); % Edge 2 -> 1
    \draw [->] (v3) to (v2); % Edge 3 -> 2
    \draw [->] (v4) to (v3); % Edge 4 -> 3
    \draw[->] (v0) edge[loop above] (v0); % Self-loop at 0
\end{tikzpicture}
}
\end{center}
Run the SageMath script \texttt{ex1325.sage} to verify.
\end{example}
\begin{proposition}[Complementary Labeling Symmetry]\label{prop:complementary-labeling-Symmetry}
%\lean{}
  %\leanok
  %\uses{defn:graceful-functional-graphs, defn:graceful-functional-graphs-set, prop:multivariate-quotient-remainder, defn:polynomial-grace-definition, defn:polynomial-congruence, prop:polynomial-grace-certificate, prop:complementary-labeling-Symmetry}
Let $f\in\mathbb{Z}_{n}^{\mathbb{Z}_{n}}$ and the remainder of $P_f$ be
\begin{equation}
\overline{P}_{f}(\mathbf{x}):=\prod_{v\in\mathbb{Z}_{n}}(v!)^{2}\frac{(n-1+v)!}{(2v)!}\sum_{\begin{array}{c}
\substack{\sigma\in\text{S}_{n}\\
\text{s.t.}\\
\left|\sigma f-\sigma\right|\in\text{S}_{n}
}
\end{array}}\text{sgn}(\sigma\left|\sigma f-\sigma\right|)\prod_{\begin{array}{c}
\substack{i\in\mathbb{Z}_{n}\\
j_{i}\in\mathbb{Z}_{n}\backslash\{\sigma(i)\}
}
\end{array}}\big(\frac{x_{i}-j_{i}}{\sigma(i)-j_{i}}\big).
\end{equation}
The complementary labeling map $x_{i}\mapsto x_{n-1-i},$ for all $i\in \mathbb{Z}_n$,
fixes $\overline{P}_f$ up to sign.
\end{proposition}
\begin{proof}
For notational convenience, let $\mathbf{x}_{\varphi}:=(x_{\varphi(0)},\ldots,x_{\varphi(i)},\ldots,x_{\varphi(n-1)})$. Observe that for any permutation $\varphi\in \text{S}_n$, the action of $\varphi$ on $P_f$ yields equalities
 \[
 \begin{array}{ccc}
P_{f}({\bf x}_{\varphi}) & = & \underset{0\le u<v<n}{\prod}(x_{\varphi(v)}-x_{\varphi(u)})\big((x_{\varphi f(v)}-x_{\varphi(v)})^{2}-(x_{\varphi f(u)}-x_{\varphi(u)})^{2}\big),\\
\\
 & = & \underset{0\le\varphi^{-1}(i)<\varphi^{-1}(j)<n}{\prod}(x_{j}-x_{i})\big((x_{\varphi f\varphi^{-1}(j)}-x_{j})^{2}-(x_{\varphi f\varphi^{-1}(i)}-x_{i})^{2}\big).
\end{array}
 \]
 The last equality above features the indexing change of variable $u=\varphi^{-1}(i)$ and $v=\varphi^{-1}(j)$. If $\varphi \in \text{Aut}\left(G_f\right)$ then $P_{f}(x_{\varphi(0)},\ldots,x_{\varphi(n-1)})$ is up to sign equal to $P_{\varphi f\varphi^{-1}}$, in accordance with Definition \ref{defn:polynomial-grace-definition}. Furthermore, by the proof of Proposition \ref{prop:polynomial-grace-certificate}, the action of $\varphi$
on $P_{f}$ yields the congruence identity
\[
P_{f}(\mathbf{x}_{\varphi})\equiv \overline{P}_f(\mathbf{x}_{\varphi}).
\]
Hence,
\begin{align*}\overline{P}_{f}({\bf x}_{\varphi}) & =\prod_{v\in\mathbb{Z}_{n}}\big((v!)^{2}\frac{(n-1+v)!}{(2v)!}\big)\sum_{\begin{array}{c}
\substack{\sigma\in\text{S}_{n}\\
\text{s.t.}\\
\left|\sigma f-\sigma\right|\in\text{S}_{n}
}
\end{array}}\text{sgn}(\sigma\left|\sigma f-\sigma\right|)\prod_{\substack{i\in\mathbb{Z}_{n}\\
j_{i}\in\mathbb{Z}_{n}\backslash\{\sigma(i)\}
}
}\bigg(\frac{x_{\varphi(i)}-j_{i}}{\sigma(i)-j_{i}}\bigg),\\
 & =\text{sgn}(\varphi)\prod_{v\in\mathbb{Z}_{n}}\bigg(\left(v!\right)^{2}\frac{\left(n-1+v\right)!}{\left(2v\right)!}\bigg)\sum_{\begin{array}{c}
\substack{\sigma\in\text{S}_{n}\\
\text{s.t.}\\
\left|\sigma f-\sigma\right|\in\text{S}_{n}
}
\end{array}}\text{sgn}(\sigma\left|\sigma f-\sigma\right|\varphi^{-1})\prod_{\substack{u\in\mathbb{Z}_{n}\\
v_{u}\in\mathbb{Z}_{n}\backslash\{\sigma\varphi^{-1}(u)\}
}
}\bigg(\frac{x_{u}-v_{u}}{\sigma\varphi^{-1}(u)-v_{u}}\bigg).
\end{align*}
If $\varphi=n-1-\text{id}$, then,
by the complementary labeling symmetry, we have 
\[
G_{\sigma f\sigma^{-1}}\in\text{GrL}\left(G_{f}\right)\Longleftrightarrow G_{\sigma\varphi^{-1}f(\sigma\varphi^{-1})^{-1}}\in\text{GrL}\left(G_{f}\right)
\]
Let $\mathfrak{G}$ denote the subrgoup of $\text{S}_n$ whose members are $\big\{\text{id},\ \varphi\big\}$. We write
\[
\overline{P}_f(\mathbf{x}_\varphi)=
\]
\[
\prod_{v\in\mathbb{Z}_{n}}\big((v!)^{2}\frac{(n-1+v)!}{(2v)!}\big)\sum_{\begin{array}{c}
\substack{\sigma\in\nicefrac{\text{S}_{n}}{\mathfrak{G}}\\
\gamma=\left|\sigma f-\sigma\right|\in\text{S}_{n}
}
\end{array}}\text{sgn}(\sigma\gamma)\,\bigg(\text{sgn}(\varphi^{-1})\prod_{\substack{u\in\mathbb{Z}_{n}\\
v_{u}\in\mathbb{Z}_{n}\backslash\{\sigma(u)\}
}
}\big(\frac{x_{u}-v_{u}}{\sigma(u)-v_{u}}\big)+\prod_{\substack{u\in\mathbb{Z}_{n}\\
v_{u}\in\mathbb{Z}_{n}\backslash\{\sigma\varphi^{-1}(u)\}
}
}\big(\frac{x_{u}-v_{u}}{\sigma\varphi^{-1}(u)-v_{u}}\big)\bigg).
\]
Similarly, 
\[
\overline{P}_f(\mathbf{x})=
\]
\[
\prod_{v\in\mathbb{Z}_{n}}\big((v!)^{2}\frac{(n-1+v)!}{(2v)!}\big)\sum_{\begin{array}{c}
\substack{\sigma\in\nicefrac{\text{S}_{n}}{\mathfrak{G}}\\
\gamma=\left|\sigma f-\sigma\right|\in\text{S}_{n}
}
\end{array}}\text{sgn}(\sigma\gamma)\,\bigg(\prod_{\begin{array}{c}
\substack{i\in\mathbb{Z}_{n}\\
j_{i}\in\mathbb{Z}_{n}\backslash\{\sigma(i)\}
}
\end{array}}\big(\frac{x_{i}-j_{i}}{\sigma(i)-j_{i}}\big)+\text{sgn}(\varphi^{-1})\prod_{\begin{array}{c}
\substack{i\in\mathbb{Z}_{n}\\
j_{i}\in\mathbb{Z}_{n}\backslash\{\sigma\varphi^{-1}(i)\}
}
\end{array}}\big(\frac{x_{i}-j_{i}}{\sigma\varphi^{-1}(i)-j_{i}}\big)\bigg).
\]
We conclude that the complementary labeling symmetry yields the equality
\[
\overline{P}_f(\mathbf{x})=\text{sgn}(\varphi)\,\overline{P}_f(\mathbf{x}_{\varphi})=\overline{P}_{\varphi f\varphi^{-1}}(\mathbf{x}),
\]
thus establishing the desired claim.
\end{proof}
\begin{example} We present an example of a path on 5 vertices.
\ \\
\begin{center}   
\resizebox{!}{3em}{
\begin{tikzpicture}[>={Stealth[round]}, node distance=2cm and 1cm, every node/.style={circle, draw, minimum size=1cm}]
    % Define vertices
    \node (v0) at (0, 0) {0};
    \node (v1) at (3, 0) {1};
    \node (v2) at (6, 0) {2};
    \node (v3) at (9, 0) {3};
    \node (v4) at (12, 0) {4};
    
    % Draw edges
    \draw [->] (v1) to (v0); % Edge 1 -> 0
    \draw [->] (v2) to (v1); % Edge 2 -> 1
    \draw [->] (v3) to (v2); % Edge 3 -> 2
    \draw [->] (v4) to (v3); % Edge 4 -> 3
    \draw[->] (v0) edge[loop above] (v0); % Self-loop at 0
\end{tikzpicture}
}
\end{center}
Run the SageMath script \texttt{ex1328.sage} to verify.
\end{example}

\section{The Composition Lemma}

\begin{lemma}[Transposition Invariance]\label{lem:transposition-invariance}
  %\lean{}
  %\leanok
  %\uses{defn:graceful-functional-graphs, defn:graceful-functional-graphs-set, prop:multivariate-quotient-remainder, defn:polynomial-grace-definition, defn:polynomial-congruence, prop:polynomial-grace-certificate, lem:transposition-invariance, lem:variable-dependency}
Let $f\in\mathbb{Z}_{n}^{\mathbb{Z}_{n}}$ be such that its functional directed graph $G_{f}$ has at least two sibling leaf nodes, i.e., $G_f$ has vertices $u,v\in\mathbb{Z}_{n}$ such that
$f^{-1}\big(\{u,v\}\big)=\varnothing$ and $f(u)=f(v)$. If
the transposition $\tau\in\textrm{S}_{n}$ exchanges $u$ and
$v$, i.e.,
\[
\tau(i)=\begin{cases}
\begin{array}{cc}
v & \text{ if }i=u\\
u & \text{ if }i=v\\
i & \text{otherwise}
\end{array} & \forall\,i\in\mathbb{Z}_{n}.\end{cases}
\]
Then
\begin{equation}
\tau\in\text{Aut}\;(P_{f}({\bf x})),
\end{equation}
where $P_f$ is the polynomial certificate of grace as defined in \ref{defn:polynomial-grace-definition}.
\end{lemma}
\begin{proof}
  %\leanok
  Stated otherwise, the claim asserts that the polynomial
$P_{f}$ is fixed by a transposition of any pair of variables associated
with sibling leaf vertices. By construction of $P_{f}\left({\bf x}\right)$,
the changes in its Vandermonde factors induced by the action of $\tau$
are as follows:
\begin{equation}
    P_{f}(x_{\tau(0)},\ldots,x_{\tau(i)},\ldots,x_{\tau(n-1)})=\prod_{0\le i<j<n}(x_{\tau(j)}-x_{\tau(i)})\prod_{0\le i<j<n}\big((x_{\tau f(j)}-x_{\tau(j)})^{2}-(x_{\tau f(i)}-x_{\tau(i)})^{2}\big).
\end{equation}
Note that there is a bijection
\begin{equation}
x_{i}\mapsto(x_{f(i)}-x_{i})^{2},\quad\forall\:i\in\mathbb{Z}_{n}.
\end{equation} Hence, the transposition $\tau \in \text{Aut}\left(G_f\right)$ of the leaf nodes induces a transposition
$\tau$ of the corresponding leaf edges outgoing from the said leaf
nodes. 
More precisely, the maps
\[
\left(\begin{array}{ccccc}
x_{0} & ,\ldots, & x_{i} & ,\ldots, & x_{n-1}\\
\downarrow &  & \downarrow &  & \downarrow\\
x_{\tau(0)} & ,\ldots, & x_{\tau(i)} & ,\ldots, & x_{\tau(n-1)}
\end{array}\right)
\]
and
\[
\left(\begin{array}{ccccc}
(x_{f(0)}-x_{0})^{2} & ,\ldots, & (x_{f(i)}-x_{i})^{2} & ,\ldots, & (x_{f(n-1)}-x_{n-1})^{2}\\
\downarrow &  & \downarrow &  & \downarrow\\
(x_{\tau f(0)}-x_{\tau(0)})^{2} & ,\ldots, & (x_{\tau f(i)}-x_{\tau(i)})^{2} & ,\ldots, & (x_{\tau f(n-1)}-x_{\tau(n-1)})^{2}
\end{array}\right)
\]
prescribe the same permutation $\tau$ of the vertex variables and induced edges label binomials respectively. Observe that
\[
P_{f}(x_{\tau(0)},\ldots,x_{\tau(i)},\ldots,x_{\tau(n-1)})=
\]
\[
\left(\prod_{0\le i<j<n}\frac{x_{\tau(j)}-x_{\tau(i)}}{x_{j}-x_{i}}\prod_{0\le i<j<n}(x_{j}-x_{i})\right)\left(\prod_{0\le i<j<n}\frac{(x_{\tau f(j)}-x_{\tau(j)})^{2}-(x_{\tau f(i)}-x_{\tau(i)})^{2}}{(x_{f(j)}-x_{j})^{2}-(x_{f(i)}-x_{i})^{2}}\prod_{0\le i<j<n}\big((x_{f(j)}-x_{j})^{2}-(x_{f(i)}-x_{i})^{2}\big)\right)
\]
\[
=\bigg(\text{sgn}(\tau)\prod_{0\le i<j<n}(x_{j}-x_{i})\bigg)\bigg(\text{sgn}(\tau)\prod_{0\le i<j<n}\big((x_{f(j)}-x_{j})^{2}-(x_{f(i)}-x_{i})^{2}\big)\bigg)
\]
\[
=\bigg((-1)\prod_{0\le i<j<n}(x_{j}-x_{i})\bigg)\bigg((-1)\prod_{0\le i<j<n}\big((x_{f(j)}-x_{j})^{2}-(x_{f(i)}-x_{i})^{2}\big)\bigg)
\]
\begin{equation}
\implies P_{f}(x_{\tau(0)},\ldots,x_{\tau(n-1)})=P_{f}(x_{0},\ldots,x_{n-1}),
\end{equation}
thus establishing the desired claim.

\end{proof}

\begin{prop}[Composition Inequality]\label{prop:composition-lemma-ineq}
  %\lean{}
  %\leanok
  %\uses{defn:graceful-functional-graphs, defn:graceful-functional-graphs-set, prop:multivariate-quotient-remainder, defn:polynomial-grace-definition, defn:polynomial-congruence, prop:polynomial-grace-certificate}
Consider an arbitrary $f\in\mathbb{Z}_{n}^{\mathbb{Z}_{n}}$ subject to the fixed point condition
$\left|f^{\left(n-1\right)}\left(\mathbb{Z}_{n}\right)\right|=1$. The following statements are equivalent:
\begin{enumerate}
    \item 
    \[
    \max_{\sigma\in\text{S}_{n}}\left|\left\{ \left|\sigma f^{(2)}\sigma^{-1}(i)-i\right|:i\in\mathbb{Z}_{n}\right\} \right|\le\max_{\sigma\in\text{S}_{n}}\left|\left\{ \left|\sigma f\sigma^{-1}(i)-i\right|:i\in\mathbb{Z}_{n}\right\} \right|.
    \]
    \item
    \[
    P_{f^{(2)}} ({\bf x}) \not\equiv 0  \implies  P_{f} ({\bf x}) \not\equiv 0.
    \]
    \item 
    \[
    \text{GrL}(G_f) \neq \varnothing
    \]
\end{enumerate}    
\end{prop}
\begin{proof}
If $f\in\mathbb{Z}_{n}^{\mathbb{Z}_{n}}$ is identically constant,
then $G_{f}$ is graceful. We see this from the fact that the functional
digraph of the identically zero function is gracefully labeled and the fact that
functional digraphs of identically constant functions are all isomorphic. It follows
that all functional directed graphs having diameter less than $3$ are graceful. Consequently,
all claims hold for all functional digraphs of diameter less than $3$. We now turn our attention
to functional trees of diameter greater or equal to $3$. 
It follows by definition 
\begin{equation}
n=\max_{\sigma\in\text{S}_{n}}\left|\left\{ \left|\sigma f\sigma^{-1}(i)-i\right|:i\in\mathbb{Z}_{n}\right\} \right| \iff P_{f}({\bf x})\not\equiv0 \iff \text{GrL}(G_f) \neq \varnothing.
\end{equation}
We now proceed to show (i) $\iff$ (iii). The backward claim is the simplest of the two claims. We see
that if $f$ is contractive, so too is $f^{(2)}$. Then the
assertions
\begin{equation}
n=\max_{\sigma\in\text{S}_{n}}\left|\left\{ |\sigma f^{(2)}\sigma^{-1}(i)-i|:i\in\mathbb{Z}_{n}\right\} \right| \text{ and } n=\max_{\sigma\in\text{S}_{n}}\left|\left\{ |\sigma f\sigma^{-1}(i)-i|:i\in\mathbb{Z}_{n}\right\} \right|
\end{equation}
indeed implies the inequality
\begin{equation}
\max_{\sigma\in\text{S}_{n}}\left|\left\{ |\sigma f^{(2)}\sigma^{-1}(i)-i|:i\in\mathbb{Z}_{n}\right\} \right|\le\max_{\sigma\in\text{S}_{n}}\left|\left\{ |\sigma f\sigma^{-1}(i)-i|:i\in\mathbb{Z}_{n}\right\} \right|.
\end{equation}
We now establish the forward claim by contradiction. Assume for the
sake of establishing a contradiction that for some contractive map
$f\in\mathbb{Z}_{n}^{\mathbb{Z}_{n}}$ we have 
\begin{equation}
n>\max_{\sigma\in\text{S}_{n}}\left|\left\{ |\sigma f^{(2)}\sigma^{-1}(i)-i|:i\in\mathbb{Z}_{n}\right\} \right|,
\end{equation}
for we know by the number of edges being equal to $n$ that it is impossible
 that 
\begin{equation}
n<\max_{\sigma\in\text{S}_{n}}\left|\left\{ |\sigma f^{(2)}\sigma^{-1}(i)-i|:i\in\mathbb{Z}_{n}\right\} \right|.
\end{equation}
Note that the range of $f$ is a proper subset of $\mathbb{Z}_{n}$.
By the premise that $f$ is contractive, it follows that $f^{(\left\lceil 2^{\text{lg}(n-1)}\right\rceil )}$
is identically constant and thus 
\begin{equation}
n=\max_{\sigma\in\text{S}_{n}}\left|\left\{ |\sigma f^{(\left\lceil 2^{\text{lg}(n-1)}\right\rceil )}\sigma^{-1}(i)-i|:i\in\mathbb{Z}_{n}\right\} \right|,
\end{equation}
where lg denotes the logarithm base $2$. Consequently there must
be some integer $0\le\kappa<\text{lg}(n-1)$ such that 
\begin{equation}
\max_{\sigma\in\text{S}_{n}}\left|\left\{ |\sigma f^{(\left\lceil 2^{\kappa}\right\rceil )}\sigma^{-1}(i)-i|:i\in\mathbb{Z}_{n}\right\} \right|>\max_{\sigma\in\text{S}_{n}}\left|\left\{ |\sigma f^{(\left\lceil 2^{\kappa-1}\right\rceil )}\sigma^{-1}(i)-i|:i\in\mathbb{Z}_{n}\right\} \right|.
\end{equation}
This contradicts the assertion of statement (i), thereby
establishing the backward claim. 
The exact same reasoning as above establishes (ii) $\iff$ (iii), for we have
\begin{equation}
        P_{f^{\left(\left\lceil 2^{\text{lg}(n-1)}\right\rceil \right)}} ({\bf x}) \not\equiv 0. 
\end{equation}
\end{proof}

Having assembled together the pieces required to prove our main result, we proceed to fit the pieces together to state and prove the \textit{Composition Lemma}.
\begin{lemma}[Composition Lemma]\label{lem:composition-lemma}
    %\lean{}
  %\leanok
  %\uses{defn:graceful-functional-graphs, defn:graceful-functional-graphs-set, prop:multivariate-quotient-remainder, defn:polynomial-grace-definition, defn:polynomial-congruence, prop:polynomial-grace-certificate, prop:composition-lemma-ineq}
 For all contractive  $f\in\mathbb{Z}_{n}^{\mathbb{Z}_{n}}$, i.e., subject to 
$\left|f^{\left(n-1\right)}\left(\mathbb{Z}_{n}\right)\right|=1$,
we have
\begin{equation}\label{eq:composition-lemma}
\max_{\sigma\in\text{S}_{n}}\left|\left\{ \left|\sigma f^{(2)}\sigma^{-1}(i)-i\right|:i\in\mathbb{Z}_{n}\right\} \right|\le\max_{\sigma\in\text{S}_{n}}\left|\left\{ \left|\sigma f\sigma^{-1}(i)-i\right|:i\in\mathbb{Z}_{n}\right\} \right|.\end{equation}
\end{lemma}
\begin{proof}
Owing to Proposition \ref{prop:composition-lemma-ineq}, we prove the statement by establishing 
\[
    P_{f^{(2)}} ({\bf x}) \not\equiv 0  \implies  P_{f} ({\bf x}) \not\equiv 0.
\]
For simplicity, we prove a generalization of the desired claim. Assume without loss of generality that the vertex labeled $(n-1)$ is at furthest edge distance from the root in $G_f$ (i.e. the fixed point ). Given that the diameter of $G_{f}$ is greater than $2$, we may also assume without loss of generality that  $f^{-1}\left(\{n-1\}\right)=\varnothing$
and $f^{(2)}(n-1)\ne f(n-1)$. Let the contractive
map $g\in\mathbb{Z}_{n}^{\mathbb{Z}_{n}}$ be devised from $f$ and an arbitrary nonempty subset $S\subseteq f^{-1}\big(\{f(n-1)\}\big)$ such
that 
\begin{equation}
g(i)=\begin{cases}
\begin{array}{cc}
f^{(2)}(i) & \text{ if }i\in S\\
f(i) & \text{otherwise}
\end{array},\ \forall\,i\in\mathbb{Z}_{n}.\end{cases}
\end{equation}
We show that
\begin{equation} \label{eq:composition-lemma-generalized}
P_{g} ({\bf x}) \not\equiv 0  \implies  P_{f} ({\bf x}) \not\equiv 0.
\end{equation}
Note that the assertion immediately above generalizes the composition
lemma since, $f$ is only partially
iterated. More precisely, we iterate $f$ on a subset $S$ subject to
$\varnothing\ne S\subseteq f^{-1}\left(\left\{ f(n-1)\right\} \right)\subset\mathbb{Z}_{n}$.
In turn, iterating (at most ${n \choose 2}$ times) this generalization  of the composition lemma yields
that all functional trees are graceful, which in turn implies that the \textit{Composition Lemma} as stated in Lemma \ref{eq:composition-lemma}  holds. For notational convenience, assume without loss of generality that 
\begin{equation}
f(n-1)=n-\left|f^{-1}\big(\{f(n-1)\}\big)\right|-1,\; f^{-1}\big(\{f(n-1)\}\big)=\mathbb{Z}_{n}\setminus\mathbb{Z}_{1+f(n-1)} \text{ and } S=\big\{ n-1,\,\cdots,\,n-|S|\big\}.
\end{equation}
If the conditions stated above are not met, we relabel the vertices of $G_f$ to ensure that such is indeed the case. Note that such a relabeling does not affect the property we seek to prove. It suffices to show that the claim holds when $S=\{n-1\}$. We prove the contrapositive claim
\begin{equation}\label{eq:contrapositive}
    P_{f}({\bf x})\equiv0\implies P_{g}({\bf x})\equiv0.
\end{equation}
By construction, the polynomial
\begin{equation}\label{eq:Pf_broken}
\begin{array}{c}
P_{f}(\mathbf{x})=\underset{0\le i<j<n-1}{\prod}(x_{j}-x_{i})\big((x_{f(j)}-x_{j})^{2}-(x_{f(i)}-x_{i})^{2}\big)\times\\[10pt]
\underset{u\in\mathbb{Z}_{n-1}}{\prod}(x_{n-1}-x_{u})\big((x_{f(n-1)}-x_{n-1})^{2}-(x_{f(u)}-x_{u})^{2}\big).\\[10pt]
\end{array}
\end{equation}
differs only slightly from
\begin{equation}\label{eq:Pg_broken}
\begin{array}{c}
P_{g}(\mathbf{x})=\underset{0\le i<j<n-1}{\prod}(x_{j}-x_{i})\big((x_{f(j)}-x_{j})^{2}-(x_{f(i)}-x_{i})^{2}\big)\times\\[10pt]
\underset{u\in\mathbb{Z}_{n-1}}{\prod}(x_{n-1}-x_{u})\big((x_{f^{(2)}(n-1)}-x_{n-1})^{2}-(x_{f(u)}-x_{u})^{2}\big).\\[10pt]
\end{array}
\end{equation}
We setup a variable \emph{telescoping} within each induced edge label binomial $(x_{f^{\left(2\right)}\left(v\right)}-x_{v})$
where $v\in S$
(i.e. induced edge binomials of edges outgoing from the subset of vertices where $f$ is iterated) as follows:
\[
\begin{array}{c}
\underbrace{{\color{blue}(}x_{f^{(2)}(n-1)}-x_{n-1}{\color{red})}}\\
x_{n-1}\longrightarrow x_{f^{(2)}(n-1)}
\end{array}\begin{array}{c}
=\\
\\
\end{array}\begin{array}{c}
\underbrace{{\color{red}(x_{f(n-1)}}-x_{n-1}{\color{red})}}\\
x_{n-1}\longrightarrow{\color{red}x_{f(n-1)}}
\end{array}\begin{array}{c}
+\\
\\
\end{array}\begin{array}{c}
\underbrace{{\color{blue}(}x_{f^{(2)}(v)}{\color{blue}-x_{f(n-1)}}{\color{blue})}}\\
{\color{blue}x_{f(n-1)}}\longrightarrow x_{f^{(2)}(n-1)}
\end{array}.
\]
We start by breaking the symmetry among chromatic variables to distinguish more clearly distinct canceling partners and formally define the transpositions which generate the $\Theta$-involution. We rewrite the telescoping setup as follows
\[
\begin{array}{c}
P_{g}(\mathbf{x})=V(\mathbf{x})\underset{0\le i<j<n-1}{\prod}\big((x_{f(j)}-x_{j})^{2}-(x_{f(i)}-x_{i})^{2}\big)\times\\[10pt]
\underset{\begin{array}{c}
u\in\mathbb{Z}_{n-1}\\
t\in\{0,1\}
\end{array}}{\prod}\big((x_{f^{(2)}(n-1)}{\color{blue}-x_{u,t,f(n-1)}}{\color{blue})}+{\color{red}(x_{u,t,f(n-1)}}-x_{n-1})+(-1)^{t}(x_{f(u)}-x_{u})\big).\\[10pt]
\end{array}
\]
The telescoping setup introduces  the $2(n-1)$ pair of canceling blue red variables:
\[
\bigg\{({\color{blue}-x_{u,t,f(n-1)}},{\color{red}x_{u,t,f(n-1)}}):\begin{array}{c}
u\in\mathbb{Z}_{n-1}\\
t\in\{0,1\}
\end{array}\bigg\}
\]
in the expression of $P_g$.
As such evaluations of chromatic variables play no role in evaluations of $P_{g}$. Invoking the multibinomial identity yields the multibinomial expansion
\[
P_{g}(\mathbf{x})=V(\mathbf{x})\underset{0\le i<j<n-1}{\prod}\big((x_{f(j)}-x_{j})^{2}-(x_{f(i)}-x_{i})^{2}\big)\left[\prod_{\begin{array}{c}
u\in\mathbb{Z}_{n-1}\\
t\in\{0,1\}
\end{array}}\big({\color{red}(x_{u,t,f(n-1)}}-x_{n-1})+(-1)^{t}(x_{f(u)}-x_{u})\big)\right.+
\]
\[
\left.\sum_{\substack{s_{ut}\in\{0,1\}\\
0=\prod_{u,t}s_{ut}
}
}\prod_{\begin{array}{c}
u\in\mathbb{Z}_{n-1}\\
t\in\{0,1\}
\end{array}}\big({\color{red}(x_{u,t,f(n-1)}}-x_{n-1})+(-1)^{t}(x_{f(u)}-x_{u})\big)^{s_{ut}}\,(x_{f^{(2)}(n-1)}{\color{blue}-x_{u,t,f(n-1)}}{\color{blue})}^{1-s_{ut}}\right].
\]
The $\Theta$-involution action on both expressions of $P_{g}$ is generated by the following $2(n-1)$ transpositions:
\[
\begin{array}{ccccc}
(x_{f^{(2)}(n-1)}{\color{blue}-x_{u,t,f(n-1)}}{\color{blue})} & \begin{array}{c}
{\color{red}\rightarrow}\\
{\color{blue}\leftarrow}
\end{array} & {\color{red}(x_{u,t,f(n-1)}}-x_{n-1})+(-1)^{t}(x_{f(u)}-x_{u}) &  & \forall\;(u,t)\in\mathbb{Z}_{n-1}\times\{0,1\}.\end{array}
\]
Each transposition exchanges a “blue” binomial (so named because it features a single blue variable ${\color{blue}-x_{u,t,f(n-1)}}$) with a corresponding complementary “red” quatrinomial partner (so named because it features the corresponding red canceling partner variable ${\color{red}x_{u,t,f(n-1)}}$). For the sake of the argument we restrict chromatic variables via assignments prescribed for all $(u,t)\in\mathbb{Z}_{n-1}\times\{0,1\}$ by
\[
{\color{red}x_{u,t,f(n-1)}}={\color{red}x_{f(n-1)}}\;\text{ and }\;{\color{blue}x_{u,t,f(n-1)}}={\color{blue}x_{f(n-1)}}.
\]
In the present setting the $2(n-1)$ transpositions which generate the $\Theta$-involution become
\[
\begin{array}{ccccc}
(x_{f^{(2)}(n-1)}{\color{blue}-x_{f(n-1)}}{\color{blue})} & \begin{array}{c}
{\color{red}\rightarrow}\\
{\color{blue}\leftarrow}
\end{array} & {\color{red}(x_{f(n-1)}}-x_{n-1})+(-1)^{t}(x_{f(u)}-x_{u}) &  & \forall\:(u,t)\in\mathbb{Z}_{n-1}\times\{0,1\}\end{array}.
\]
The ensuing telescoping setup and multibinomial expansion of are respectively
\[
\begin{array}{c}
P_{g}(\mathbf{x})=V(\mathbf{x})\underset{0\le i<j<n-1}{\prod}\big((x_{f(j)}-x_{j})^{2}-(x_{f(i)}-x_{i})^{2}\big)\times\\[10pt]
\underset{(u,t)\in\mathbb{Z}_{n-1}\times\{0,1\}}{\prod}\big((x_{f^{(2)}(n-1)}{\color{blue}-x_{f(n-1)}}{\color{blue})}+{\color{red}(x_{f(n-1)}}-x_{n-1})+(-1)^{t}(x_{f(u)}-x_{u})\big)\\[10pt]
\end{array}
\]
and
\[
P_{g}(\mathbf{x})=V(\mathbf{x})\underset{0\le i<j<n-1}{\prod}\big((x_{f(j)}-x_{j})^{2}-(x_{f(i)}-x_{i})^{2}\big)\left[\prod_{\begin{array}{c}
u\in\mathbb{Z}_{n-1}\\
t\in\{0,1\}
\end{array}}\big({\color{red}(x_{f(n-1)}}-x_{n-1})+(-1)^{t}(x_{f(u)}-x_{u})\big)\right.+
\]
\[
\left.\sum_{\substack{s_{ut}\in\{0,1\}\\
0=\prod_{u,t}s_{ut}
}
}\prod_{\begin{array}{c}
u\in\mathbb{Z}_{n-1}\\
t\in\{0,1\}
\end{array}}\big({\color{red}(x_{f(n-1)}}-x_{n-1})+(-1)^{t}(x_{f(u)}-x_{u})\big)^{s_{ut}}\,(x_{f^{(2)}(n-1)}{\color{blue}-x_{f(n-1)}}{\color{blue})}^{1-s_{ut}}\right].
\]
We now proceed with the \emph{chromatic argument}. Let $R_{f,g}$ denote the polynomial which results from the removal of the monochromatic red summand from the multibinomial expansion of $P_{g}$. Namely
\[
\begin{array}{c}
R_{f,g}(\mathbf{x})=V(\mathbf{x})\underset{0\le i<j<n-1}{\prod}\big((x_{f(j)}-x_{j})^{2}-(x_{f(i)}-x_{i})^{2}\big)\times\\[10pt]
\underset{\substack{s_{ut}\in\{0,1\}\\
0=\prod_{u,t}s_{ut}
}
}{\sum}\underset{(u,t)\in\mathbb{Z}_{n-1}\times\{0,1\}}{\prod}\big({\color{red}(x_{f(n-1)}}-x_{n-1})+(-1)^{t}(x_{f(u)}-x_{u})\big)^{s_{ut}}\,(x_{f^{(2)}(n-1)}{\color{blue}-x_{f(n-1)}}{\color{blue})}^{1-s_{ut}}.\\[10pt]
\end{array}
\]
In the multibinomial expansion of $P_{g}$, the $\Theta$-involution transposes the unique monochromatic red summand given by
\[
{\color{red}P_{f}(\mathbf{x})}=V(\mathbf{x})\underset{0\le i<j<n-1}{\prod}\big((x_{f(j)}-x_{j})^{2}-(x_{f(i)}-x_{i})^{2}\big)\prod_{(u,t)\in\mathbb{Z}_{n-1}\times\{0,1\}}\big({\color{red}(x_{f(n-1)}}-x_{n-1})+(-1)^{t}(x_{f(u)}-x_{u})\big),
\]
with the unique monochromatic blue summand given by
\[
V(\mathbf{x})\underset{0\le i<j<n-1}{\prod}\big((x_{f(j)}-x_{j})^{2}-(x_{f(i)}-x_{i})^{2}\big)\prod_{(u,t)\in\mathbb{Z}_{n-1}\times\{0,1\}}(x_{f^{(2)}(n-1)}{\color{blue}-x_{f(n-1)}}{\color{blue})}.
\]
Furthermore the $\Theta$-involution transposes each multibinomial bichromatic summand of the form
\[
V(\mathbf{x})\underset{0\le i<j<n-1}{\prod}\big((x_{f(j)}-x_{j})^{2}-(x_{f(i)}-x_{i})^{2}\big)\prod_{\begin{array}{c}
u\in\mathbb{Z}_{n-1}\\
t\in\{0,1\}
\end{array}}\big({\color{red}(x_{f(n-1)}}-x_{n-1})+(-1)^{t}(x_{f(u)}-x_{u})\big)^{s_{ut}}(x_{f^{(2)}(n-1)}{\color{blue}-x_{f(n-1)}}{\color{blue})}^{1-s_{ut}},
\]
such that
\[
s_{ut}\in\{0,1\}\text{ where }(u,t)\in\mathbb{Z}_{n-1}\times\{0,1\}\text{ and }\bigg(\prod_{(u,t)\in\mathbb{Z}_{n-1}\times\{0,1\}}s_{ut}\bigg)=0\ne\bigg(\sum_{(u,t)\in\mathbb{Z}_{n-1}\times\{0,1\}}s_{ut}\bigg),
\]
with the corresponding complementary multibinomial bichromatic summand given by
\[
V(\mathbf{x})\underset{0\le i<j<n-1}{\prod}\big((x_{f(j)}-x_{j})^{2}-(x_{f(i)}-x_{i})^{2}\big)\prod_{\begin{array}{c}
u\in\mathbb{Z}_{n-1}\\
t\in\{0,1\}
\end{array}}\big({\color{red}(x_{f(n-1)}}-x_{n-1})+(-1)^{t}(x_{f(u)}-x_{u})\big)^{1-s_{u,t}}\,(x_{f^{(2)}(n-1)}{\color{blue}-x_{f(n-1)}}{\color{blue})}^{s_{ut}}.
\]
By construction $R_{f,g}$ is not fixed by the $\Theta$-involution for it lacks the monochromatic red summand but features the monochromatic blue summand. Indeed the $\Theta$-involution maps $R_{f,g}$ to the polynomial which results from the removal of the monochromatic blue summand from the multibinomial expansion of $P_{g}$. In other words the $\Theta$-involution maps $R_{f,g}$ to
\[
V(\mathbf{x})\underset{0\le i<j<n-1}{\prod}\big((x_{f(j)}-x_{j})^{2}-(x_{f(i)}-x_{i})^{2}\big)\left[\prod_{\begin{array}{c}
u\in\mathbb{Z}_{n-1}\\
t\in\{0,1\}
\end{array}}\big({\color{red}(x_{f(n-1)}}-x_{n-1})+(-1)^{t}(x_{f(u)}-x_{u})\big)\right.+
\]
\[
\left.\sum_{\substack{s_{ut}\in\{0,1\}\\
(\underset{u,t}{\prod}s_{ut})=0\ne(\underset{u,t}{\sum}s_{ut})
}
}\prod_{\begin{array}{c}
u\in\mathbb{Z}_{n-1}\\
t\in\{0,1\}
\end{array}}\big({\color{red}(x_{f(n-1)}}-x_{n-1})+(-1)^{t}(x_{f(u)}-x_{u})\big)^{s_{ut}}\,(x_{f^{(2)}(n-1)}{\color{blue}-x_{f(n-1)}}{\color{blue})}^{1-s_{ut}}\right].
\]
In particular chromatic variables no longer cancel each other out in  $R_{f,g}$. We now describe the action induced by the $\Theta$-involution on the multibinomial expansions of $\overline{P}_{g}$ and $\overline{R}_{f,g}$ expressed with respect to the Lagrange basis. We start with congruence identities
\[
\overline{P}_{g}(\mathbf{x})\equiv
\]
\[
\sum_{\sigma\in\text{S}_{n}}\bigg(V(\sigma)\prod_{0\le i<j<n-1}\big((\sigma f(j)-\sigma(j))^{2}-(\sigma f(i)-\sigma(i))^{2}\big)L_{\sigma}\bigg)\left[\prod_{\begin{array}{c}
u\in\mathbb{Z}_{n-1}\\
t\in\{0,1\}
\end{array}}\bigg(\big({\color{red}(\sigma f(n-1)}-\sigma(n-1))+(-1)^{t}\big(\sigma f(u)-\sigma(u)\big)\big){\color{red}L_{\sigma}}\bigg)\right.+
\]
\[
\left.\sum_{\substack{s_{ut}\in\{0,1\}\\
0=\prod_{u,t}s_{ut}
}
}\prod_{\begin{array}{c}
u\in\mathbb{Z}_{n-1}\\
t\in\{0,1\}
\end{array}}\bigg(\big({\color{red}(\sigma f(n-1)}-\sigma(n-1))+(-1)^{t}\big(\sigma f(u)-\sigma(u)\big)\big)\,{\color{red}L_{\sigma}}\bigg)^{s_{ut}}\bigg(\big(\sigma f^{(2)}(n-1){\color{blue}-\sigma f(n-1)}{\color{blue}\big)}{\color{blue}L_{\sigma}}\bigg)^{1-s_{ut}}\right].
\]
and
\[
\overline{R}_{f,g}(\mathbf{x})\equiv\sum_{\sigma\in\text{S}_{n}}\bigg(V(\sigma)\prod_{0\le i<j<n-1}\big((\sigma f(j)-\sigma(j))^{2}-(\sigma f(i)-\sigma(i))^{2}\big)\,L_{\sigma}\bigg)\times
\]
\[
\left[\sum_{\substack{s_{ut}\in\{0,1\}\\
0=\prod_{u,t}s_{ut}
}
}\prod_{\begin{array}{c}
u\in\mathbb{Z}_{n-1}\\
t\in\{0,1\}
\end{array}}\bigg({\color{red}\big(}{\color{red}\sigma f(n-1)}-\sigma(n-1)\big)+(-1)^{t}\big(\sigma f(u)-\sigma(u)\big)\big)\,{\color{red}L_{\sigma}}\bigg)^{s_{ut}}\bigg(\big(\sigma f^{(2)}(n-1){\color{blue}-\sigma f(n-1)}{\color{blue}\big)}\,{\color{blue}L_{\sigma}}\bigg)^{1-s_{ut}}\right].
\]
The action induced by the $\Theta$-involution on the two expressions immediately above is generated by transpositions of scaled Lagrange basis summand associated with each $\sigma\in\text{S}_{n}$ and $(u,t)\in\mathbb{Z}_{n-1}\times\{0,1\}$
\[
\big(\sigma f^{(2)}(n-1){\color{blue}-\sigma f(n-1)}{\color{blue}\big)}\,{\color{blue}L_{\sigma}}(x_{0},\ldots,{\color{blue}x_{f(n-1)}},x_{n-1})\begin{array}{c}
{\color{red}\rightarrow}\\
{\color{blue}\leftarrow}
\end{array}{\color{red}\big(}{\color{red}\sigma f(n-1)}-\sigma(n-1)\big)+(-1)^{t}\big(\sigma f(u)-\sigma(u)\big)\big)\,{\color{red}L_{\sigma}}(x_{0},\ldots,{\color{red}x_{f(n-1)}},x_{n-1}).
\]
The latter transpositions of scaled Lagrange bases summands indexed by members of $S_{n}$ which generate the induced $\Theta$-involution are devised from the $2(n-1)$ generators:
\[
\begin{array}{ccccc}
(x_{f^{(2)}(n-1)}{\color{blue}-x_{f(n-1)}}{\color{blue})} & \begin{array}{c}
{\color{red}\rightarrow}\\
{\color{blue}\leftarrow}
\end{array} & {\color{red}(x_{f(n-1)}}-x_{n-1})+(-1)^{t}(x_{f(u)}-x_{u}) &  & \forall\:(u,t)\in\mathbb{Z}_{n-1}\times\{0,1\}.\end{array}
\]
taken with equalities
\[
(x_{f^{(2)}(n-1)}{\color{blue}-x_{f(n-1)}}{\color{blue})}=\sum_{h\in\mathbb{Z}_{n}^{\mathbb{Z}_{n}}}\big(hf^{(2)}(n-1){\color{blue}-hf(n-1)}{\color{blue}\big)}\,{\color{blue}L_{h}}(x_{0},x_{1},\ldots,{\color{blue}x_{f(n-1)}},x_{n-1}),
\]
and for all $(u,t)\in\mathbb{Z}_{n-1}\times\{0,1\}$
\[
{\color{red}(x_{f(n-1)}}-x_{n-1})+(-1)^{t}(x_{f(u)}-x_{u})=\sum_{h\in\mathbb{Z}_{n}^{\mathbb{Z}_{n}}}{\color{red}\big(}{\color{red}hf(n-1)}-h(n-1)\big)+(-1)^{t}\big(hf(u)-h(u)\big)\big)\,{\color{red}L_{h}}(x_{0},x_{1},\ldots,{\color{red}x_{f(n-1)}},x_{n-1}).
\]
We invoke the orthonormality property of the Lagrange basis as well the following rules based upon Proposition \ref{prop:Orthogonality}, 
\[
\begin{cases}
\begin{array}{ccc}
L_{\sigma}(x_{0},x_{1},\cdots,x_{f(n-1)},x_{n-1})\cdot{\color{red}L_{\sigma}}(x_{0},x_{1},\ldots,{\color{red}x_{f(n-1)}},x_{n-1}) & \text{is replaced by} & {\color{red}L_{\sigma}}(x_{0},x_{1},\ldots,{\color{red}x_{f(n-1)}},x_{n-1}),\\
\\
L_{\sigma}(x_{0},x_{1},\cdots,x_{f(n-1)},x_{n-1})\cdot{\color{blue}L_{\sigma}}(x_{0},x_{1},\ldots,{\color{blue}x_{f(n-1)}},x_{n-1}) & \text{is replaced by} & {\color{blue}L_{\sigma}}(x_{0},x_{1},\ldots,{\color{blue}x_{f(n-1)}},x_{n-1}),\\
\\
L_{\sigma}(x_{0},x_{1},\cdots,x_{f(n-1)},x_{n-1})\cdot{\color{violet}L_{\sigma}}(x_{0},x_{1},\ldots,{\color{violet}x_{f(n-1)}},x_{n-1}) & \text{is replaced by} & {\color{violet}L_{\sigma}}(x_{0},x_{1},\ldots,{\color{violet}x_{f(n-1)}},x_{n-1}),\\
\\
{\color{red}L_{\sigma}}(x_{0},x_{1},\ldots,{\color{red}x_{f(n-1)}},x_{n-1})\cdot{\color{blue}L_{\sigma}}(x_{0},x_{1},\ldots,{\color{blue}x_{f(n-1)}},x_{n-1}) & \text{is replaced by} & {\color{violet}L_{\sigma}}(x_{0},x_{1},\ldots,{\color{violet}x_{f(n-1)}},x_{n-1}),\\
\\
{\color{blue}L_{\sigma}}(x_{0},x_{1},\ldots,{\color{blue}x_{f(n-1)}},x_{n-1})\cdot{\color{violet}L_{\sigma}}(x_{0},x_{1},\ldots,{\color{violet}x_{f(n-1)}},x_{n-1}) & \text{is replaced by} & {\color{violet}L_{\sigma}}(x_{0},x_{1},\ldots,{\color{violet}x_{f(n-1)}},x_{n-1}),\\
\\
{\color{red}L_{\sigma}}(x_{0},x_{1},\ldots,{\color{red}x_{f(n-1)}},x_{n-1})\cdot{\color{violet}L_{\sigma}}(x_{0},x_{1},\ldots,{\color{violet}x_{f(n-1)}},x_{n-1}) & \text{is replaced by} & {\color{violet}L_{\sigma}}(x_{0},x_{1},\ldots,{\color{violet}x_{f(n-1)}},x_{n-1}).\\
\\
{\color{red}L_{\sigma}}(x_{0},x_{1},\ldots,{\color{red}x_{f(n-1)}},x_{n-1})\cdot{\color{red}L_{\sigma}}(x_{0},x_{1},\ldots,{\color{red}x_{f(n-1)}},x_{n-1}) & \text{is replaced by} & {\color{red}L_{\sigma}}(x_{0},x_{1},\ldots,{\color{red}x_{f(n-1)}},x_{n-1}),\\
\\
{\color{blue}L_{\sigma}}(x_{0},x_{1},\ldots,{\color{blue}x_{f(n-1)}},x_{n-1})\cdot{\color{blue}L_{\sigma}}(x_{0},x_{1},\ldots,{\color{blue}x_{f(n-1)}},x_{n-1}) & \text{is replaced by} & {\color{blue}L_{\sigma}}(x_{0},x_{1},\ldots,{\color{blue}x_{f(n-1)}},x_{n-1}),\\
\\
{\color{violet}L_{\sigma}}(x_{0},x_{1},\ldots,{\color{violet}x_{f(n-1)}},x_{n-1})\cdot{\color{violet}L_{\sigma}}(x_{0},x_{1},\ldots,{\color{violet}x_{f(n-1)}},x_{n-1}) & \text{is replaced by} & {\color{violet}L_{\sigma}}(x_{0},x_{1},\ldots,{\color{violet}x_{f(n-1)}},x_{n-1}).
\end{array}\end{cases}
\]
Rules immediately above enable us to keep track of occurrences and colors of chromatic variables featured in multibinomial summands expressed with respect to the Lagrange basis. By their repeated use, we devise chromatic multibinomial expansions for $\overline{P}_{g}$ and $\overline{R}_{f,g}$ respectively given by
\[
\overline{P}_{g}=\sum_{\sigma\in\text{S}_{n}}V(\sigma)\prod_{0\le i<j<n-1}\big((\sigma f(j)-\sigma(j))^{2}-(\sigma f(i)-\sigma(i))^{2}\big)\left[\prod_{\begin{array}{c}
u\in\mathbb{Z}_{n-1}\\
t\in\{0,1\}
\end{array}}(\sigma f^{(2)}(n-1){\color{blue}-\sigma f(n-1)}{\color{blue})}\,{\color{blue}L_{\sigma}}+\right.
\]
\[
\sum_{\substack{s_{ut}\in\{0,1\}\\
(\underset{u,t}{\prod}s_{ut})=0\ne(\underset{u,t}{\sum}s_{ut})
}
}\prod_{\begin{array}{c}
u\in\mathbb{Z}_{n-1}\\
t\in\{0,1\}
\end{array}}{\color{red}\bigg(\sigma f(n-1)}-\sigma(n-1)+(-1)^{t}\big(\sigma f(u)-\sigma(u)\big)\bigg)^{s_{ut}}\big(\sigma f^{(2)}(n-1){\color{blue}-\sigma f(n-1)}{\color{blue}\big)}^{1-s_{ut}}\,{\color{violet}L_{\sigma}}+
\]
\[
\left.\prod_{\begin{array}{c}
u\in\mathbb{Z}_{n-1}\\
t\in\{0,1\}
\end{array}}{\color{red}\bigg(\sigma f(n-1)}-\sigma(n-1)+(-1)^{t}\big(\sigma f(u)-\sigma(u)\big)\bigg)\,{\color{red}L_{\sigma}}\right],
\]
and 
\[
\overline{R}_{f,g}=\sum_{\sigma\in\text{S}_{n}}V(\sigma)\prod_{0\le i<j<n-1}\big((\sigma f(j)-\sigma(j))^{2}-(\sigma f(i)-\sigma(i))^{2}\big)\left[\prod_{\begin{array}{c}
u\in\mathbb{Z}_{n-1}\\
t\in\{0,1\}
\end{array}}(\sigma f^{(2)}(n-1){\color{blue}-\sigma f(n-1)}{\color{blue})}\,{\color{blue}L_{\sigma}}+\right.
\]
\[
\left.\sum_{\substack{s_{ut}\in\{0,1\}\\
(\underset{u,t}{\prod}s_{ut})=0\ne(\underset{u,t}{\sum}s_{ut})
}
}\prod_{\begin{array}{c}
u\in\mathbb{Z}_{n-1}\\
t\in\{0,1\}
\end{array}}{\color{red}\bigg(\sigma f(n-1)}-\sigma(n-1)+(-1)^{t}\big(\sigma f(u)-\sigma(u)\big)\bigg)^{s_{ut}}\big(\sigma f^{(2)}(n-1){\color{blue}-\sigma f(n-1)}{\color{blue}\big)}^{1-s_{ut}}\,{\color{violet}L_{\sigma}}\right].
\]
In both expansions immediately above, for each permutation $\sigma\in\text{S}_{n}$, scaled Lagrange basis summands featured can be partitioned according to their colors.
For each permutation $\sigma\in\text{S}_{n}$, the corresponding summands in the expansion feature three distinct types of chromatic scaled Lagrange basis summands. The first type is a scaling of a red Lagrange basis element defined as
\[
{\color{red}L_{\sigma}}(x_{0},x_{1},\ldots,{\color{red}x_{f(n-1)}},x_{n-1}):=\prod_{\begin{array}{c}
j_{f(n-1)}\in\mathbb{Z}_{n}\backslash\{\sigma f(n-1)\}\end{array}}\big(\frac{{\color{red}x_{f(n-1)}}-j_{f(n-1)}}{\sigma f(n-1)-j_{f(n-1)}}\big)\,\prod_{i\in\mathbb{Z}_{n}\backslash\{f(n-1)\}}\bigg(\prod_{j_{i}\in\mathbb{Z}_{n}\backslash\{\sigma(i)\}}\big(\frac{x_{i}-j_{i}}{\sigma(i)-j_{i}}\big)\bigg),
\]
which is scaled by a corresponding red coefficient (so named because it features the evaluation of red variable) of the precise form
\[
{\color{red}P_{f}}(\sigma)=V(\sigma)\prod_{0\le i<j<n-1}\big((\sigma f(j)-\sigma(j))^{2}-(\sigma f(i)-\sigma(i))^{2}\big)\prod_{\begin{array}{c}
u\in\mathbb{Z}_{n-1}\\
t\in\{0,1\}
\end{array}}\bigg({\color{red}(\sigma f(n-1)}-\sigma(n-1))+(-1)^{t}\big(\sigma f(u)-\sigma(u)\big)\bigg).
\]
The second type of scaled chromatic Lagrange basis summand is  a scaling of a blue Lagrange basis element defined as
\[
{\color{blue}L_{\sigma}}(x_{0},x_{1},\ldots,{\color{blue}x_{f(n-1)}},x_{n-1}):=\prod_{\begin{array}{c}
j_{f(n-1)}\in\mathbb{Z}_{n}\backslash\{\sigma f(n-1)\}\end{array}}\big(\frac{{\color{blue}x_{f(n-1)}}-j_{f(n-1)}}{\sigma f(n-1)-j_{f(n-1)}}\big)\,\prod_{i\in\mathbb{Z}_{n}\backslash\{f(n-1)\}}\bigg(\prod_{j_{i}\in\mathbb{Z}_{n}\backslash\{\sigma(i)\}}\big(\frac{x_{i}-j_{i}}{\sigma(i)-j_{i}}\big)\bigg),
\]
which is scaled by the corresponding blue coefficient (so named because it features the evaluation of blue variable) of the precise form
\[
V(\sigma)\prod_{0\le i<j<n-1}\big((\sigma f(j)-\sigma(j))^{2}-(\sigma f(i)-\sigma(i))^{2}\big)\prod_{(u,t)\in\mathbb{Z}_{n-1}\times\{0,1\}}(\sigma f^{(2)}(n-1){\color{blue}-\sigma f(n-1)}{\color{blue})}.
\]
The third and last type of scaled chromatic Lagrange basis summand is a scaling of a purple Lagrange basis element defined as
\[
{\color{violet}L_{\sigma}}(x_{0},x_{1},\ldots,{\color{violet}x_{f(n-1)}},x_{n-1}):=\prod_{\begin{array}{c}
j_{f(n-1)}\in\mathbb{Z}_{n}\backslash\{\sigma f(n-1)\}\end{array}}\big(\frac{{\color{violet}x_{f(n-1)}}-j_{f(n-1)}}{\sigma f(n-1)-j_{f(n-1)}}\big)\,\prod_{i\in\mathbb{Z}_{n}\backslash\{f(n-1)\}}\bigg(\prod_{j_{i}\in\mathbb{Z}_{n}\backslash\{\sigma(i)\}}\big(\frac{x_{i}-j_{i}}{\sigma(i)-j_{i}}\big)\bigg),
\]
which is scaled by the corresponding (red and blue) bichromatic coefficient (so named because it features the evaluation of both a red and a blue variable) of the precise form
\[
\prod_{(u,t)\in\mathbb{Z}_{n-1}\times\{0,1\}}\bigg({\color{red}\sigma f(n-1)}-\sigma(n-1)+(-1)^{t}\big(\sigma f(u)-\sigma(u)\big)\bigg)^{s_{ut}}(\sigma f^{(2)}(n-1){\color{blue}-\sigma f(n-1)}{\color{blue})}^{1-s_{ut}},
\]
such that
\[
s_{ut}\in\{0,1\}\text{ where }(u,t)\in\mathbb{Z}_{n-1}\times\{0,1\}\text{ and }\bigg(\prod_{(u,t)\in\mathbb{Z}_{n-1}\times\{0,1\}}s_{ut}\bigg)=0\ne\bigg(\sum_{(u,t)\in\mathbb{Z}_{n-1}\times\{0,1\}}s_{ut}\bigg).
\]
For every $\sigma\in\text{S}_{n}$ the involutive action induced by the $\Theta$-involution maps the scaled red Lagrange basis summand
\[
{\color{red}P_{f}}(\sigma)\,{\color{red}L_{\sigma}}(x_{0},x_{1},\ldots,{\color{red}x_{f(n-1)}},x_{n-1})
\]
 to the blue scaled Lagrange basis summand
 \[
 V(\sigma)\prod_{0\le i<j<n-1}\big((\sigma f(j)-\sigma(j))^{2}-(\sigma f(i)-\sigma(i))^{2}\big)\prod_{(u,t)\in\mathbb{Z}_{n-1}\times\{0,1\}}\big(\sigma f^{(2)}(n-1){\color{blue}-\sigma f(n-1)}{\color{blue}\big)}\,{\color{blue}L_{\sigma}}(x_{0},x_{1},\ldots,{\color{blue}x_{f(n-1)}},x_{n-1})
 \]
 and vice versa. Furthermore, the involutive action induced by the $\Theta$-involution maps each scaled purple scaled Lagrange basis summand of the form
 \[
 V(\sigma)\prod_{0\le i<j<n-1}\big((\sigma f(j)-\sigma(j))^{2}-(\sigma f(i)-\sigma(i))^{2}\big)\times
 \]
 \[
 \prod_{(u,t)\in\mathbb{Z}_{n-1}\times\{0,1\}}\bigg({\color{red}\sigma f(n-1)}-\sigma(n-1)+(-1)^{t}\big(\sigma f(u)-\sigma(u)\big)\bigg)^{s_{ut}}(\sigma f^{(2)}(n-1){\color{blue}-\sigma f(n-1)}{\color{blue})}^{1-s_{ut}}\,{\color{violet}L_{\sigma}}(x_{0},x_{1},\ldots,{\color{violet}x_{f(n-1)}},x_{n-1}),
 \]
 to a corresponding purple scaled Lagrange basis summand
 \[
 V(\sigma)\prod_{0\le i<j<n-1}\big((\sigma f(j)-\sigma(j))^{2}-(\sigma f(i)-\sigma(i))^{2}\big)\times
 \]
 \[
 \prod_{(u,t)\in\mathbb{Z}_{n-1}\times\{0,1\}}\bigg({\color{red}\sigma f(n-1)}-\sigma(n-1)+(-1)^{t}\big(\sigma f(u)-\sigma(u)\big)\bigg)^{1-s_{ut}}(\sigma f^{(2)}(n-1){\color{blue}-\sigma f(n-1)}{\color{blue})}^{s_{ut}}\,{\color{violet}L_{\sigma}}(x_{0},x_{1},\ldots,{\color{violet}x_{f(n-1)}},x_{n-1}),
 \]
 and vice versa. In summary, the involutive action induced by the $\Theta$-involution pairs up (by the involution map) purple scaled Lagrange basis summands among themselves while simultaneously pairs up each red scaled Lagrange basis summand with a corresponding blue scaled Lagrange basis summand and vice versa. By our premise, the $\Theta$-involution symmetry holds for $\overline{P}_{g}=\overline{R}_{f,g}$. Given by our premise that coefficients of red Lagrange basis summands all vanish i.e. ${\color{red}P_{f}}(\sigma)=0$ for all $\sigma\in\text{S}_{n}$ a necessary condition for the $\Theta$-involution symmetry is that
 \[
 V(\sigma)\prod_{0\le i<j<n-1}\big((\sigma f(j)-\sigma(j))^{2}-(\sigma f(i)-\sigma(i))^{2}\big)\prod_{(u,t)\in\mathbb{Z}_{n-1}\times\{0,1\}}(\sigma f^{(2)}(n-1){\color{blue}-\sigma f(n-1)}{\color{blue})}=0,
 \]
 for all $\sigma\in\text{S}_{n}$. However the premise $P_{g}\not\equiv0$ implies
 \[
 V(\sigma)\prod_{0\le i<j<n-1}\big((\sigma f(j)-\sigma(j))^{2}-(\sigma f(i)-\sigma(i))^{2}\big)\prod_{(u,t)\in\mathbb{Z}_{n-1}\times\{0,1\}}(\sigma f^{(2)}(n-1){\color{blue}-\sigma f(n-1)}{\color{blue})}\ne0,
 \]
 for all $\sigma\in\text{S}_{n}$ subject to $G_{\sigma g\sigma^{-1}}\in\text{GrL}\big(G_{g}\big)$. This fact constitutes an obstruction to the $\Theta$-involution symmetry. Indeed owing to the fact that chromatic variables should play no role in evaluations of $\overline{P}_{g}$ we know that for each permutation $\sigma\in\text{S}_{n}$ the linear combination of chromatic evaluations featured in the coefficients of the corresponding Lagrange basis vanish. However in the multibinomial expansion expressing $\overline{R}_{f,g}$ , for each permutation $\sigma\in\text{S}_{n}$ the linear combination of chromatic evaluations featured in coefficients of Lagrange basis elements indexed by $\sigma$ is equal to
 \[
 -V(\sigma)\prod_{0\le i<j<n-1}\big((\sigma f(j)-\sigma(j))^{2}-(\sigma f(i)-\sigma(i))^{2}\big)\prod_{\begin{array}{c}
u\in\mathbb{Z}_{n-1}\\
t\in\{0,1\}
\end{array}}\big({\color{red}\sigma f(n-1)}\big).
 \]
 We call the quantity immediately above the chromatic defect. The chromatic defect is the left over after cancellation of evaluation of telescoping variables featured in the coefficient of the Lagrange basis indexed by $\sigma$ . Since for each permutation $\sigma$ the corresponding linear combinations of evaluations of chromatic variables which features as coefficient of the Lagrange basis index by $\sigma\in\text{S}_{n}$ in the multibinomial expansion of $\overline{P}_g$ is given by
 \[
 0=V(\sigma)\prod_{0\le i<j<n-1}\bigg(\big(\sigma f(j)-\sigma(j)\big)^{2}-\big(\sigma f(i)-\sigma(i)\big)^{2}\bigg)\times
 \]
 \[
 \bigg[\prod_{\begin{array}{c}
u\in\mathbb{Z}_{n-1}\\
t\in\{0,1\}
\end{array}}\big({\color{red}\sigma f(n-1)}\big)+\sum_{\substack{s_{ut}\in\{0,1\}\\
0=\prod_{u,t}s_{ut}
}
}\prod_{\begin{array}{c}
u\in\mathbb{Z}_{n-1}\\
t\in\{0,1\}
\end{array}}\big({\color{red}\sigma f(n-1)}\big)^{s_{ut}}\big({\color{blue}-\sigma f(n-1)}\big)^{1-s_{ut}}\bigg].
 \]
 Roughly speaking the chromatic defect quantifies for a given permutation $\sigma\in\text{S}_{n}$ the associated $\Theta$-involution symmetry breaking. For the $\Theta$-involution symmetry to hold the chromatic defect must vanish. However by the complementary labeling involution symmetry there must be at least one permutation $\sigma\in\{\gamma,\,n-1-\gamma\}$ for which $\sigma f(n-1)\ne0$ whenever $G_{\gamma g\gamma^{-1}}\in\text{GrL}\big(G_{g}\big)$. Thus we established that the chromatic defect is non-vanishing which in turn breaks the $\Theta$-involution symmetry and afford a non-trivial role to evaluations of chromatic variables when evaluating $R_{f,g}$ over $\mathbb{Z}_{n}^{\mathbb{Z}_{n}}$. By which we conclude that $P_{g}\not\equiv R_{f,g}\implies P_{f}\not\equiv0$ as claimed.
\end{proof}

\begin{example} We present a verification of Lemma \ref{lem:composition-lemma} with an example of a path on 5 vertices.
\ \\
\begin{center}   $G_f: \;$
\resizebox{!}{3em}{
\begin{tikzpicture}[>={Stealth[round]}, node distance=2cm and 1cm, every node/.style={circle, draw, minimum size=1cm}]
    % Define vertices
    \node (v0) at (0, 0) {0};
    \node (v1) at (3, 0) {1};
    \node (v2) at (6, 0) {2};
    \node (v3) at (9, 0) {3};
    \node (v4) at (12, 0) {4};
    
    % Draw edges
    \draw [->] (v1) to (v0); % Edge 1 -> 0
    \draw [->] (v2) to (v1); % Edge 2 -> 1
    \draw [->] (v3) to (v2); % Edge 3 -> 2
    \draw [->] (v4) to (v3); % Edge 4 -> 3
    \draw[->] (v0) edge[loop above] (v0); % Self-loop at 0
\end{tikzpicture}
}
\end{center}
\ \\
\begin{center}   
$G_g: \;$
\resizebox{!}{3em}{
\begin{tikzpicture}[>={Stealth[round]}, node distance=2cm and 1cm, every node/.style={circle, draw, minimum size=1cm}]
    % Define vertices
    \node (v0) at (0, 0) {0};
    \node (v1) at (3, 0) {1};
    \node (v2) at (6, 0) {2};
    \node (v3) at (9, 0) {3};
    \node (v4) at (12, 0) {4};
    
    % Draw edges
    \draw [->] (v1) to (v0); % Edge 1 -> 0
    \draw [->] (v2) to (v1); % Edge 2 -> 1
    \draw [->] (v3) to (v2); % Edge 3 -> 2
    \draw [->] (v4) edge[bend right] (v2); % Edge 4 -> 2
    \draw[->] (v0) edge[loop above] (v0); % Self-loop at 0
\end{tikzpicture}
}
\end{center}
Run the SageMath script \texttt{ex1434.sage} to verify.
\end{example}