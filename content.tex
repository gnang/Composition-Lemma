`% In this file you should put the actual content of the blueprint.
% It will be used both by the web and the print version.
% It should *not* include the \begin{document}
%
% If you want to split the blueprint content into several files then
% the current file can be a simple sequence of \input. Otherwise It
% can start with a \section or \chapter for instance.

\chapter{\texorpdfstring{Composition Lemma}{Composition Lemma}}

\section{Overview}
The \textit{Composition Lemma} was developed and refined over 6 years, beginning in 2018, as a novel approach to settle in the affirmative the \textit{Graceful Tree Conjecture}. The first of such papers was posted in \cite{gnang2020gracefullabelingstrees} by Gnang. A further developed series of papers resolving the same conjecture again appeared in \cite{gnang2022compositionlemma} and \cite{gnang2023proofkotzigringelrosaconjecture}. Recently, the same method has been applied to settle other longstanding conjectures in \cite{chalise2024treenedgesdecomposes} and \cite{chalise2024prooftreepackingconjecture}. We comment that the series of papers shared on the open-source platform arXiv reflect the evolving landscape of Gnang's thought process, and the frequent re-uploads were driven by the natural progression and refinement of ideas. However, we recognize that these numerous edits may have unintentionally caused confusion and raised questions regarding the success of the method. In the current work, we aim to address these concerns by presenting a detailed blueprint of the proof, with the goal of formalizing it in Lean4.

\section{Functional Directed Graphs}\label{sec:Functional Directed Graphs}
For notational convenience, let $\mathbb{Z}_{n}$ denote the set whose members are the smallest $n$ non-negative integers, i.e.,
\begin{equation}
\mathbb{Z}_{n}:=\big\{0,\ldots,\,n-1\big\}.
\end{equation}
For a function $f:\mathbb{Z}_{m}\to\mathbb{Z}_{n}$, we write $f\in\mathbb{Z}_{n}^{\mathbb{Z}_{m}}$.
For $X\subseteq\mathbb{Z}_{m}$, $f(X)$ denotes the image of $X$
under $f$, i.e.,
\begin{equation}
f(X)=\{f(i):i\in X\},
\end{equation}
and $|f(X)|$ denotes its
cardinality. For $Y\subseteq\mathbb{Z}_{n}\ensuremath{,}f^{-1}(Y)$
denotes the pre-image of $Y$ under $f$ i.e.
\begin{equation}
f^{-1}(Y)=\{j\in\mathbb{Z}_{m}:f(j)\in Y\}
\end{equation}
\begin{definition}[Functional digraphs]\label{defn:functional-directed-graphs}
% REVIEW THE LEAN LABELS BELLOW:
  %\lean{}
  %\leanok
  %\uses{defn:standard-simplex}
For an arbitrary $f\in \mathbb{Z}_n^{\mathbb{Z}_n}$, the \emph{functional directed graph} prescribed by $f$, denoted $G_f$, is such that the vertex set $V(G_f)$ and the directed edge set $E(G_f)$ are respectively as follows:
\[
V(G_f) = \mathbb{Z}_n, \; E(G_f) = \{(v,f(v)):v \in \mathbb{Z}_n\}.
\]
\end{definition}
\begin{definition}[Graceful functional digraphs]\label{defn:graceful-functional-graphs}
  %\lean{}
  %\leanok
  %\uses{defn:functional-directed-graphs}
The functional directed graph prescribed by $f\in\mathbb{Z}_{n}^{\mathbb{Z}_{n}}$ is graceful if there exist
a bijection $\sigma\in \text{S}_n \subset
 \mathbb{Z}_{n}^{\mathbb{Z}_{n}}$ such that
\begin{equation}
\big\{\left|\sigma f\sigma^{-1}(i)-i\right|:i\in\mathbb{Z}_{n}\big\}=\mathbb{Z}_{n}.
\end{equation}
If $\sigma=\text{id}$ (the identity function), then $G_{f}$ --- the functional directed graph prescribed by $f$ --- is gracefully labeled. 
\end{definition}
\begin{defn}[Automorphism group]\label{defn:aut-functional-graphs}
  %\lean{}
  %\leanok
  %\uses{defn:functional-directed-graphs}
For a functional directed graph $G_f$, its automorphism group, denoted $\text{Aut}\left(G_f\right)$, is defined as follows:
\[
\text{Aut}(G_{f})=\big\{\sigma\in\text{S}_{n}:\{\big(i,f(i)\big):i\in\mathbb{Z}_{n}\}=\{\big(j,\sigma f\sigma^{-1}(j)\big):j\in\mathbb{Z}_{n}\}\big\}.
\]
For a polynomial $P \in \mathbb{C}[x_0, \ldots, x_{n-1}]$, its automorphism group, is the stablizer of $P$ and denoted $\text{Aut($P$)}$. Formally defined as follows:
\[
\text{Aut}\left(P\right)=\large\{\sigma\in\text{S}_{n}:P\left(x_{0},\ldots,x_{i},\ldots,x_{n-1}\right)=P\left(x_{\sigma(0)},\ldots,x_{\sigma(i)},\ldots,x_{\sigma(n-1)}\right)\large\}.
\]
\end{defn}
\begin{definition}[Graceful re-labelings]\label{defn:graceful-functional-graphs-set}
  %\lean{}
  %\leanok
  %\uses{defn:functional-directed-graphs, defn:graceful-functional-graphs}
The set of distinct gracefully labeled functional directed graphs isomorphic to $G_{f}$ is 
\[
\text{GrL}(G_{f})\,:=\left\{ G_{\sigma f\sigma^{-1}}:\begin{array}{c}
\sigma\text{ is a representative of a coset in }\nicefrac{\text{S}_{n}}{\text{Aut}(G_{f})}\text{ and }\\
\mathbb{Z}_{n}=\big\{|\sigma f\sigma^{-1}\left(i\right)-i|\,:\,i\in\mathbb{Z}_{n}\big\}
\end{array}\right\} 
\]
\end{definition}
\begin{definition}[Complementary labeling involution]\label{defn:complementary-labeling-symmetry}
  %\lean{}
  %\leanok
  %\uses{defn:functional-directed-graphs, defn:graceful-functional-graphs}
If $\varphi=n-1-\text{id}$, i.e. $\varphi \in \mathbb{Z}_{n}^{\mathbb{Z}_{n}}$ such that
\[
\varphi(i)=n-1-i,\, \forall \,i\in \mathbb{Z}_{n},
\]
The complementary labeling involution is defined as the map whose domain and codomain is $\mathbb{Z}_{n}^{\mathbb{Z}_{n}}$ and is prescribed by
\[
f \mapsto \varphi f \varphi^{-1},
\]
for an arbitrary $f\in\mathbb{Z}_{n}^{\mathbb{Z}_{n}}$.
\end{definition}
Observe that for all $f\in \mathbb{Z}_{n}^{\mathbb{Z}_{n}}$ the complementary labeling involution fixes the induced edge label of each edge as seen from the equality
\begin{equation}
\left|f(i)-i\right|=\left|\varphi f(i)-\varphi(i)\right|,\quad\forall\,i\in\mathbb{Z}_{n}.
\end{equation}
In other words, induced edge labels are fixed by the vertex relabeling effected by $\varphi$. We call this induced edge label symmetry the \emph{complementary labeling symmetry} of the functional directed graph $G_f$.

\section{Quotient-Remainder Theorem and Lagrange Interpolation}\label{sec:QRT}
 \begin{proposition}[Multivariate Quotient-Remainder]\label{prop:multivariate-quotient-remainder}
  %\lean{Algebra.Polynomial.FieldDivision}
  %\leanok
  %\uses{}
Let $d(x)\in\mathbb{C}[x]$ be a degree $n$ monic polynomial with
simple roots, i.e.,
\begin{equation}
d(x)=\prod_{i\in\mathbb{Z}_{n}}(x-\alpha_{i})\;\text{ and }\:0\ne\prod_{0\le u<v<n}(\alpha_{v}-\alpha_{u}),
\end{equation}
where $\{\alpha_{u}:u\in\mathbb{Z}_{n}\}\subset\mathbb{C}$. 
For
all $P\in\mathbb{C}[x_{0},\ldots,x_{m-1}]$, there exists a unique remainder
$r(x_{0},\ldots,x_{m-1})\in\mathbb{C}[x_{0},\ldots,x_{m-1}]$ of degree
at most $n-1$ in each variable such that for quotients: $\big\{ q_{k}(x_{0},\ldots,x_{n-1}):k\in\mathbb{Z}_{n}\big\}\subset\mathbb{C}[x_{0},\ldots,x_{n-1}]$, we have
\begin{equation}
P(x_{0},\ldots,x_{m-1})= r(x_{0},\ldots,x_{m-1})+\sum_{u\in\mathbb{Z}_{m}}q_{u}(x_{0},\ldots,x_{m-1})\,d(x_{u}).
\end{equation}
 \end{proposition}
\begin{proof}
  %\leanok
We prove by induction on the number of variables that the remainder admits the expansion
\begin{equation}
r(x_{0},\ldots,x_{m-1})=\sum_{g\in\mathbb{Z}_{n}^{\mathbb{Z}_{m}}}P(\boldsymbol{\alpha}_{g})\prod_{i\in\mathbb{Z}_{m}}\left(\prod_{j_{i}\in\mathbb{Z}_{n}\backslash\{g(i)\}}\bigg(\frac{x_{i}-\alpha_{j_{i}}}{\alpha_{g(i)}-\alpha_{j_{i}}}\bigg)\right),
\end{equation}
where for notational convenience $P(\alpha_{g}):=P(\alpha_{g(0)},\ldots,\alpha_{g(m-1)})$.
The base case stems from the univariate quotient-remainder theorem over the field $\mathbb{C}$. The univariate-quotient remainder theorem over the field $\mathbb{C}$ asserts that there exist
a unique quotient-remainder pair $\big(q(x_{0}),\,r(x_{0})\big)\in\mathbb{C}[x_{0}]\times\mathbb{C}[x_{0}]$
subject to
\begin{equation}
H(x_{0})=q(x_{0})\,d(x_{0})+r(x_{0}),
\end{equation}
where $r(x_{0})\in\mathbb{C}[x_{0}]$ is of degree at most $n-1$.
It is completely determined by its evaluation over $\{\alpha_{i}:i\in\mathbb{Z}_{n}\}$,
and by Lagrange interpolation we have 
\begin{equation}
r(x_{0})=\sum_{g\in\mathbb{Z}_{n}^{\mathbb{Z}_{1}}}H(\alpha_{g(0)})\prod_{j_{0}\in\mathbb{Z}_{n}\backslash\{g(0)\}}\bigg(\frac{x_{0}-\alpha_{j_{0}}}{\alpha_{g(0)}-\alpha_{j_{0}}}\bigg),
\end{equation}
thus establishing the claim in the base case. For the induction step,
assume as our induction hypothesis that for all $F\in\mathbb{C}\left[x_{0},\ldots,x_{m-1}\right]$,
we have
\begin{equation}
F=\sum_{k\in\mathbb{Z}_{m}}q_{k}(x_{0},\ldots,x_{m-1})\,d(x_{k})+\sum_{g\in\mathbb{Z}_{n}^{\mathbb{Z}_{m}}}F(\alpha_{g})\prod_{i\in\mathbb{Z}_{m}}\left(\prod_{j_{i}\in\mathbb{Z}_{n}\backslash\left\{ g(i)\right\} }\left(\frac{x_{i}-\alpha_{j_{i}}}{\alpha_{g(i)}-\alpha_{j_{i}}}\right)\right).
\end{equation}
We proceed to show that the hypothesis implies that every polynomial in
$m+1$ variables also admits a similar expansion, thus establishing
the desired claim. Consider a polynomial $H \in \mathbb{C}[x_{0},\ldots,x_{m}]$. We view $H$ as a univariate polynomial in the variable $x_{m}$ whose coefficients lie in the field of fraction $\mathbb{C}(x_{0},\ldots,x_{m-1})$.
The univariate quotient-remainder theorem over the field of fractions $\mathbb{C}(x_{0},\ldots,x_{m-1})$ asserts that
there exit a unique quotient-remainder pair
\[
\big(q(x_{m}),\,r(x_{m})\big)\in\big(\mathbb{C}(x_{0},\ldots,x_{m-1})\big)[x_{m}]\times\big(\mathbb{C}(x_{0},\ldots,x_{m-1})\big)[x_{m}]
\]
subject to
\begin{equation}
H\big(x_{0},\ldots,x_{m}\big)=q(x_{0},\ldots,x_{m})\,d(x_{m})+r(x_{0},\ldots,x_{m}),
\end{equation}
where $r\left(x_{0},\ldots,x_{m}\right)\in\big(\mathbb{C}(x_{0},\ldots,x_{m-1})\big)[x_{m}]$
is of degree at most $n-1$ in the variable $x_{m}$. We write 
\begin{equation}
r\left(x_{0},\ldots,x_{m}\right)=\sum_{k\in\mathbb{Z}_{n}}a_{k}\left(x_{0},\ldots,x_{m-1}\right)\,(x_{m})^{k}.
\end{equation}
We now show that coefficients $\big\{ a_k(x_{0},\ldots,x_{m-1}):k\in \mathbb{Z}_n \big\}$ all lie in the polynomial ring $\mathbb{C}[x_{0},\ldots,x_{m-1}]$
via the equality 
\begin{equation}
\bigg(\text{Vander}\left(\begin{array}{c}
\alpha_{0}\\
\vdots\\
\alpha_{u}\\
\vdots\\
\alpha_{n-1}
\end{array}\right)\bigg)\cdot\left(\begin{array}{c}
a_{0}\left(x_{0},\ldots,x_{m-1}\right)\\
\vdots\\
a_{u}\left(x_{0},\ldots,x_{m-1}\right)\\
\vdots\\
a_{n-1}\left(x_{0},\ldots,x_{m-1}\right)
\end{array}\right)=\left(\begin{array}{c}
H(x_{0},\ldots,x_{m-1},\alpha_{0})\\
\vdots\\
H(x_{0},\ldots,x_{m-1},\alpha_{u})\\
\vdots\\
H(x_{0},\ldots,x_{m-1},\alpha_{n-1})
\end{array}\right),
\end{equation}
where 
\begin{equation}
\bigg(\text{Vander}\left(\begin{array}{c}
\alpha_{0}\\
\vdots\\
\alpha_{u}\\
\vdots\\
\alpha_{u}
\end{array}\right)\bigg)\left[i,j\right]=(\alpha_{i})^{j},\ \forall\,0\le i,j<n.
\end{equation}
Since the Vandermonde matrix is invertible by the fact
\begin{equation}
0\ne\det\bigg(\text{Vander}\left(\begin{array}{c}
\alpha_{0}\\
\vdots\\
\alpha_{u}\\
\vdots\\
\alpha_{u}
\end{array}\right)\bigg)=\prod_{0\le u<v<n}(\alpha_{v}-\alpha_{u}),
 \end{equation}
we indeed have 
\begin{equation}
\left(\begin{array}{c}
a_{0}\left(x_{0},\ldots,x_{m-1}\right)\\
\vdots\\
a_{u}\left(x_{0},\ldots,x_{m-1}\right)\\
\vdots\\
a_{n-1}\left(x_{0},\ldots,x_{m-1}\right)
\end{array}\right)=\bigg(\text{Vander}\left(\begin{array}{c}
\alpha_{0}\\
\vdots\\
\alpha_{u}\\
\vdots\\
\alpha_{u}
\end{array}\right)\bigg)^{-1}\cdot\left(\begin{array}{c}
H(x_{0},\ldots,x_{m-1},\alpha_{0})\\
\vdots\\
H(x_{0},\ldots,x_{m-1},\alpha_{u})\\
\vdots\\
H(x_{0},\ldots,x_{m-1},\alpha_{n-1})
\end{array}\right).
\end{equation}
Therefore, we have 
\begin{equation}
H\big(x_{0},\ldots,x_{m}\big)
= q_{m}\big(x_{0},\ldots,x_{m}\big)\,d(x_{m})+\sum_{g(m)\in\mathbb{Z}_{n}}H\big(x_{0},\ldots,x_{m-1},\alpha_{g(m)}\big)\prod_{j\in\mathbb{Z}_{n}\backslash\{g(m)\}}\left(\frac{x_{m}-\alpha_{j_{m}}}{\alpha_{g(m)}-\alpha_{j_{m}}}\right).
\end{equation}
Applying the induction hypothesis to coefficients
\[
\left\{ H\left(x_{0},\ldots,x_{m-1},\alpha_{g\left(m\right)}\right): \alpha_{g\left(m\right)}\in\mathbb{C}\right\} \subset\mathbb{C}[x_{0},\ldots,x_{m-1}]
\]
yields the desired expansion. Finally, quotients $\big\{ q_{k}(x_{0},\ldots,x_{m-1}):k\in\mathbb{Z}_{m}\big\}$ lie in the polynomial ring $\mathbb{C}[x_{0},\ldots,x_{m-1}]$ since the polynomial $H(x_0,\ldots,x_{m-1})-r(x_0,\ldots,x_{m-1})$ lies in the ideal generated by members of the set $\big\{ d(x_{u}):u\in\mathbb{Z}_{m}\big\}$.
\end{proof}
\begin{proposition}[Ring Homomorphism]\label{prop:ring-homomorphism}
  %\lean{Algebra.Polynomial.FieldDivision}
  %\leanok
  %\uses{}
For an arbitrary $H\in\mathbb{C}\left[x_{0},\ldots,x_{n-1}\right]$, let $\overline{H}$ denote
the {remainder} of the congruence class 
\[
H\text{ modulo the ideal generated by }\left\{ d(x_{i}):i\in\mathbb{Z}_{n}\right\} ,
\]
where
\[
d(x)=\prod_{i\in\mathbb{Z}_{n}}(x-\alpha_{i})\;\text{ and }\:0\ne\prod_{0\le u<v<n}(\alpha_{v}-\alpha_{u}),
\]
Then the following hold:
\begin{enumerate}
    \item For all $g\in\mathbb{Z}_{n}^{\mathbb{Z}_{n}}$, we have $\overline{H}(\boldsymbol{\alpha}_g)=H(\boldsymbol{\alpha}_g)$. 
    \item If $H = H_0 + H_1,$ where $H_0, H_1 \in \mathbb{C}\left[x_{0},\ldots,x_{n-1}\right]$, then $\overline{H_0} + \overline{H_1} = \overline{H}$.
    \item If $H = H_0 \cdot H_1,$ where $H_0, H_1 \in \mathbb{C}\left[x_{0},\ldots,x_{n-1}\right]$,  then $ \overline{H} \equiv \overline{H_0} \cdot \overline{H_1}$.
\end{enumerate}
 \end{proposition}
\begin{proof}
  %\leanok
  The first claim follows from Proposition \ref{prop:multivariate-quotient-remainder} for we see that the divisor vanishes over the lattice. To prove the second claim we recall that
  \[
\overline{H}=\sum_{g\in\mathbb{Z}_{n}^{\mathbb{Z}_{n}}}H(\boldsymbol{\alpha}_{g})\prod_{i\in\mathbb{Z}_{n}}\left(\prod_{j_{i}\in\mathbb{Z}_{n}\backslash\{g(i)\}}\bigg(\frac{x_{i}-\alpha_{j_{i}}}{\alpha_{g(i)}-\alpha_{j_{i}}}\bigg)\right),
  \]
  \[
  \implies\overline{H}=\sum_{g\in\mathbb{Z}_{n}^{\mathbb{Z}_{n}}}\big(H_{0}(\boldsymbol{\alpha}_{g})+H_{1}(\boldsymbol{\alpha}_{g})\big)\prod_{i\in\mathbb{Z}_{n}}\left(\prod_{j_{i}\in\mathbb{Z}_{n}\backslash\{g(i)\}}\bigg(\frac{x_{i}-\alpha_{j_{i}}}{\alpha_{g(i)}-\alpha_{j_{i}}}\bigg)\right),
  \]
  \[
  \implies\overline{H}=\sum_{k\in\mathbb{Z}_{2}}\sum_{g\in\mathbb{Z}_{n}^{\mathbb{Z}_{n}}}H_{k}(\boldsymbol{\alpha}_{g})\prod_{i\in\mathbb{Z}_{n}}\left(\prod_{j_{i}\in\mathbb{Z}_{n}\backslash\{g(i)\}}\bigg(\frac{x_{i}-\alpha_{j_{i}}}{\alpha_{g(i)}-\alpha_{j_{i}}}\bigg)\right).
  \]
  Thus $\overline{H_0} + \overline{H_1} = \overline{H}$ as claimed. Finally the fact (iii) is a straightforward consequence of Proposition \ref{prop:Orthogonality}, which is proved next.
\end{proof}
\begin{proposition}
 \label{prop:Orthogonality}
  %\lean{}
  %\leanok
  %\uses{prop:multivariate-quotient-remainder}   
 Let $f,g \in\mathbb{Z}_{n}^{\mathbb{Z}_{n}}$. For congruence classes prescribed modulo the ideal generated by $\{d(x_{i}):i \in \Z_n\}$, if
\[
d(x)=\prod_{i\in\mathbb{Z}_{n}}(x-\alpha_{i})\;\text{ such that }\:0\ne\prod_{0\le u<v<n}(\alpha_{v}-\alpha_{u}),
\]
then
\[
L_{f}({\bf x})\cdot L_{g}({\bf x})\equiv\begin{cases}
L_{f}({\bf x}) & \text{if }f=g\\
0 & \text{otherwise,}
\end{cases}
\]
\end{proposition}
\begin{proof} Observe that
\[
L_{f}(\mathbf{x})\cdot L_{g}(\mathbf{x})\;=\;\prod_{i\in\mathbb{Z}_{n}}\bigg(\big(c_{i,f}\frac{d(x_{i})}{x_{i}-\alpha_{f(i)}}\big)\big(c_{i,g}\frac{d(x_{i})}{x_{i}-\alpha_{g(i)}}\big)\bigg),
\]
where
\[
c_{i,f}=\prod_{j_{i}\in\mathbb{Z}_{n}\backslash\{f(i)\}}\big(\alpha_{f(i)}-\alpha_{j_{i}}\big)^{-1}\quad\text{ and }\quad c_{i,g}=\prod_{j_{i}\in\mathbb{Z}_{n}\backslash\{g(i)\}}\big(\alpha_{g(i)}-\alpha_{j_{i}}\big)^{-1}.
\]
If $f \neq g$, then there exists $j \in \Z_n$ such that $f(j)\neq g(j)$ and $L_{f}(\mathbf{x})\cdot L_{g}(\mathbf{x})$ is a multiple of $d(x_{j})$,
as a result of which we obtain $L_{f}(\mathbf{x})\cdot L_{g}(\mathbf{x}) \equiv 0$. Alternatively if $f = g$, then
\[
L_{f}(\mathbf{x})\cdot L_{g}(\mathbf{x})=\big(L_{f}(\mathbf{x})\big)^{2}=L_{f}(\mathbf{x})+\bigg(\big(L_{f}(\mathbf{x})\big)^{2}-L_{f}(\mathbf{x})\bigg).
\]
We now show that $\big(L_{f}(\mathbf{x})\big)^{2}-L_{f}(\mathbf{x})\equiv 0$ modulo the ideal generated by $\left\{ d(x_{i}):i\in\mathbb{Z}_{n}\right\}$.
\begin{align*}
\big(L_{f}(\mathbf{x})\big)^{2}-L_{f}(\mathbf{x})\; & =\;L_{f}(\mathbf{x})\left(L_{f}(\mathbf{x})-1\right)\\
 & =\;L_{f}(\mathbf{x})\bigg(L_{f}(\mathbf{x})-\sum_{g\in\mathbb{Z}_{n}^{\mathbb{Z}_{n}}}L_{g}\left({\bf x}\right)\bigg)\\
 & =\;-L_{f}(\mathbf{x})\bigg(\sum_{g\in\mathbb{Z}_{n}^{\mathbb{Z}_{n}}\setminus\{f\}}L_{g}\left({\bf x}\right)\bigg)\\
 & \equiv\;0,
\end{align*}
where the latter congruence identity stems from the prior setting where $f\ne g$.
\end{proof}

\begin{defn}[Polynomial of Grace]\label{defn:polynomial-grace-definition} We define $P_f \in \mathbb{C}[x_0, \ldots, x_{n-1}]$ for all $f \in \mathbb{Z}_n^{\mathbb{Z}_n}$ as follows:
\begin{equation}
    P_f(\mathbf{x}) \coloneq \underbrace{\prod_{0\le u<v<n}(x_{v}-x_{u})}_{V(x_0,\ldots,x_{n-1})}\,\underbrace{\prod_{0\le u<v<n}\big((x_{f(v)}-x_{v})^{2}-(x_{f(u)}-x_{u})^{2}\big)}_{E_f(x_0,\ldots,x_{n-1})}.
\end{equation}
\end{defn}
\begin{defn}[Congruence class]\label{defn:polynomial-congruence}
  %\lean{}  %\leanok
  %\uses{prop:multivariate-quotient-remainder}
For polynomials $P,Q \in \mathbb{C}[x_0, \ldots, x_{n-1}]$, if
\begin{equation}
P({\bf x}) \equiv \; Q({\bf x}) \; \mod\bigg\{\prod_{j\in\mathbb{Z}_{n}}(x_{i}-j):i\in\mathbb{Z}_{n}\bigg\},
\end{equation}
we simply write $P \equiv Q$.
\end{defn}
 Unless otherwise stated, all subsequent congruence identities are prescribed modulo the ideal of polynomials generated by members of the set
\[
\bigg\{\prod_{j\in\mathbb{Z}_{n}}(x_{i}-j):i\in\mathbb{Z}_{n}\bigg\}
\]
\begin{proposition}[Certificate of Grace]\label{prop:polynomial-grace-certificate}
  %\lean{}
  %\leanok
  %\uses{defn:polynomial-grace-definition, defn:polynomial-congruence, prop:multivariate-quotient-remainder}
Let $f\in\mathbb{Z}_{n}^{\mathbb{Z}_{n}}$. The functional
directed graph $G_f$ prescribed by $f$ is graceful if and only if $P_f(\mathbf{x}) \; \not\equiv \; 0$.
\end{proposition}
\begin{proof}
Observe that the vertex Vandermonde factor $V(\mathbf{x})$ is of degree exactly $n-1$ in each variable and therefore equal
to its remainder, i.e.,
\begin{equation}\label{eq:Vf}
V(\mathbf{x})=\sum_{\theta\in\text{S}_{n}}\text{sgn}(\theta)\prod_{i\in\mathbb{Z}_{n}}(x_{i})^{\theta(i)}=\prod_{v\in\mathbb{Z}_{n}}(v!)\;\sum_{\theta\in\text{S}_{n}}\text{sgn}(\theta)\prod_{\begin{array}{c}
\substack{i\in\mathbb{Z}_{n}\\
j_{i}\in\mathbb{Z}_{n}\backslash\{\theta(i)\}
}
\end{array}}\left(\frac{x_{i}-j_{i}}{\theta(i)-j_{i}}\right),
\end{equation}
where 
\begin{equation}
\text{sgn}(\theta):=\prod_{0\le u<v<n}\left(\frac{\theta(v)-\theta(u)}{v-u}\right),\quad\forall\,\theta\in\text{S}_{n}.
\end{equation}
When $n>2$, for every $f\in\mathbb{Z}_{n}^{\mathbb{Z}_{n}}$, the induced edge label Vandermonde factor $E_f(\mathbf{x})$
is of degree $>(n-1)$ in some of its variables. Therefore, by Proposition  \ref{prop:multivariate-quotient-remainder}, we have
\begin{equation}\label{eq:Ef}
E_{f}(\mathbf{x})=\sum_{l\in\mathbb{Z}_{m}}q_{l}(\mathbf{x})\prod_{k\in\mathbb{Z}_{n}}(x_{l}-k)+\prod_{v\in\mathbb{Z}_{n}}(v!)\frac{(n-1+v)!}{(2v)!}\sum_{\begin{array}{c}
g\in\mathbb{Z}_{n}^{\mathbb{Z}_{n}}\\
|gf-g|\in\text{S}_{n}
\end{array}}\text{sgn}(|gf-g|)\prod_{\begin{array}{c}
\substack{i\in\mathbb{Z}_{n}\\
j_{i}\in\mathbb{Z}_{n}\backslash\{g(i)\}
}
\end{array}}\big(\frac{x_{i}-j_{i}}{g(i)-j_{i}}\big).
\end{equation}
Observe that by the expansions in \ref{eq:Vf} and \ref{eq:Ef}, 
\[
P_{f}(\mathbf{x})=\sum_{l\in\mathbb{Z}_{m}}q_{l}(\mathbf{x})V(\mathbf{x})\prod_{k\in\mathbb{Z}_{n}}(x_{l}-k)+
\]
\[
\bigg(\prod_{v\in\mathbb{Z}_{n}}v!\sum_{\theta\in\text{S}_{n}}\text{sgn}(\theta)\prod_{\begin{array}{c}
\substack{i\in\mathbb{Z}_{n}\\
j_{i}\in\mathbb{Z}_{n}\backslash\{\theta(i)\}
}
\end{array}}\big(\frac{x_{i}-j_{i}}{\theta(i)-j_{i}}\big)\bigg)\bigg(\prod_{v\in\mathbb{Z}_{n}}(v!)\frac{(n-1+v)!}{(2v)!}\sum_{\begin{array}{c}
g\in\mathbb{Z}_{n}^{\mathbb{Z}_{n}}\\
|gf-g|\in\text{S}_{n}
\end{array}}\text{sgn}(|gf-g|)\prod_{\begin{array}{c}
\substack{i\in\mathbb{Z}_{n}\\
j_{i}\in\mathbb{Z}_{n}\backslash\{g(i)\}
}
\end{array}}\big(\frac{x_{i}-j_{i}}{g(i)-j_{i}}\big)\bigg).
\]
is congruent to
\begin{equation} \label{eq:graceful-evaluation}
\prod_{v\in\mathbb{Z}_{n}}(v!)^{2}\frac{(n-1+v)!}{(2v)!}\sum_{\begin{array}{c}
\substack{\sigma\in\text{S}_{n}\\
\text{s.t.}\\
\left|\sigma f-\sigma\right|\in\text{S}_{n}
}
\end{array}}\text{sgn}(\sigma\left|\sigma f-\sigma\right|)\prod_{\begin{array}{c}
\substack{i\in\mathbb{Z}_{n}\\
j_{i}\in\mathbb{Z}_{n}\backslash\{\sigma(i)\}
}
\end{array}}\big(\frac{x_{i}-j_{i}}{\sigma(i)-j_{i}}\big),
\end{equation}
where the permutation $|\sigma f-\sigma |$ denotes the induced edge label permutation associated with a graceful relabeling $G_{\sigma f\sigma^{-1}}$ of $G_f$.
The congruence above stems from Prop. \ref{prop:Orthogonality}. A graceful labeling necessitates the integer coefficient
\[
\prod_{0\le i<j<n}(j-i)\big(j^{2}-i^{2})=\prod_{0\le i<j<n}(j-i)^{2}(j+i)=\prod_{v\in\mathbb{Z}_{n}}\left(v!\right)^{2}\frac{\left(n-1+v\right)!}{\left(2v\right)!} \neq 0,
\]
thus establishing the desired claim.
\end{proof}
\begin{example} We present an example of a path on 5 vertices. This is known to be graceful, so we expect a non-zero remainder.
\ \\
\begin{center}   
\resizebox{!}{3em}{
\begin{tikzpicture}[>={Stealth[round]}, node distance=2cm and 1cm, every node/.style={circle, draw, minimum size=1cm}]
    % Define vertices
    \node (v0) at (0, 0) {0};
    \node (v1) at (3, 0) {1};
    \node (v2) at (6, 0) {2};
    \node (v3) at (9, 0) {3};
    \node (v4) at (12, 0) {4};
    
    % Draw edges
    \draw [->] (v1) to (v0); % Edge 1 -> 0
    \draw [->] (v2) to (v1); % Edge 2 -> 1
    \draw [->] (v3) to (v2); % Edge 3 -> 2
    \draw [->] (v4) to (v3); % Edge 4 -> 3
    \draw[->] (v0) edge[loop above] (v0); % Self-loop at 0
\end{tikzpicture}
}
\end{center}
Run the SageMath script \texttt{ex1325.sage} to verify.
\end{example}

\section{The Composition Lemma}

\begin{prop}[Composition Inequality]\label{prop:composition-lemma-ineq}
  %\lean{}
  %\leanok
  %\uses{defn:graceful-functional-graphs, defn:graceful-functional-graphs-set, prop:multivariate-quotient-remainder, defn:polynomial-grace-definition, defn:polynomial-congruence, prop:polynomial-grace-certificate}
Consider an arbitrary $f\in\mathbb{Z}_{n}^{\mathbb{Z}_{n}}$ subject to the fixed point condition
$\left|f^{\left(n-1\right)}\left(\mathbb{Z}_{n}\right)\right|=1$. The following statements are equivalent:
\begin{enumerate}
    \item 
    \[
    \max_{\sigma\in\text{S}_{n}}\big|\{|\sigma f^{(2)}\sigma^{-1}(i)-i|:i\in\mathbb{Z}_{n}\}\big|\le\max_{\sigma\in\text{S}_{n}}\big|\{|\sigma f\sigma^{-1}(i)-i|:i\in\mathbb{Z}_{n}\}\big|.
    \]
    \item
    \[
    P_{f^{(2)}} ({\bf x}) \not\equiv 0  \implies  P_{f} ({\bf x}) \not\equiv 0.
    \]
    \item 
    \[
    \text{GrL}(G_f) \neq \varnothing
    \]
\end{enumerate}    
\end{prop}
\begin{proof}
If $f\in\mathbb{Z}_{n}^{\mathbb{Z}_{n}}$ is identically constant,
then $G_{f}$ is graceful. We see this from the fact that the functional
digraph of the identically zero function is gracefully labeled and the fact that
functional digraphs of identically constant functions are all isomorphic. It follows
that all functional directed graphs having diameter less than $3$ are graceful. Consequently,
all claims hold for all functional digraphs of diameter less than $3$. We now turn our attention
to functional trees of diameter greater or equal to $3$. 
It follows by definition 
\begin{equation}
n=\max_{\sigma\in\text{S}_{n}}\big|\{|\sigma f\sigma^{-1}(i)-i|:i\in\mathbb{Z}_{n}\}\big|\iff P_{f}({\bf x})\not\equiv0\iff\text{GrL}(G_{f})\neq\varnothing.
\end{equation}
We now proceed to show (i) $\iff$ (iii). The backward claim is the simplest of the two claims. We see
that if $f$ is contractive, so too is $f^{(2)}$. Then 
assertions
\begin{equation}
n=\max_{\sigma\in\text{S}_{n}}\big|\{|\sigma f^{(2)}\sigma^{-1}(i)-i|:i\in\mathbb{Z}_{n}\}\big|\text{  and  }n=\max_{\sigma\in\text{S}_{n}}\big|\{|\sigma f\sigma^{-1}(i)-i|:i\in\mathbb{Z}_{n}\}\big|
\end{equation}
indeed implies the inequality
\begin{equation}
\max_{\sigma\in\text{S}_{n}}\big|\{|\sigma f^{(2)}\sigma^{-1}(i)-i|:i\in\mathbb{Z}_{n}\}\big|\le\max_{\sigma\in\text{S}_{n}}\big|\{|\sigma f\sigma^{-1}(i)-i|:i\in\mathbb{Z}_{n}\}\big|.
\end{equation}
We now establish the forward claim by contradiction. Assume for the
sake of establishing a contradiction that for some contractive map
$f\in\mathbb{Z}_{n}^{\mathbb{Z}_{n}}$ we have 
\begin{equation}
n>\max_{\sigma\in\text{S}_{n}}\left|\left\{ |\sigma f^{(2)}\sigma^{-1}(i)-i|:i\in\mathbb{Z}_{n}\right\} \right|,
\end{equation}
for we know by the number of edges being equal to $n$ that it is impossible
 that 
\begin{equation}
n<\max_{\sigma\in\text{S}_{n}}\left|\left\{ |\sigma f^{(2)}\sigma^{-1}(i)-i|:i\in\mathbb{Z}_{n}\right\} \right|.
\end{equation}
Note that the range of $f$ is a proper subset of $\mathbb{Z}_{n}$.
By the premise that $f$ is contractive, it follows that $f^{(\left\lceil 2^{\text{lg}(n-1)}\right\rceil )}$
is identically constant and thus 
\begin{equation}
n=\max_{\sigma\in\text{S}_{n}}\left|\left\{ |\sigma f^{(\left\lceil 2^{\text{lg}(n-1)}\right\rceil )}\sigma^{-1}(i)-i|:i\in\mathbb{Z}_{n}\right\} \right|,
\end{equation}
where lg denotes the logarithm base $2$. Consequently there must
be some integer $0\le\kappa<\text{lg}(n-1)$ such that 
\begin{equation}
\max_{\sigma\in\text{S}_{n}}\left|\left\{ |\sigma f^{(\left\lceil 2^{\kappa}\right\rceil )}\sigma^{-1}(i)-i|:i\in\mathbb{Z}_{n}\right\} \right|>\max_{\sigma\in\text{S}_{n}}\left|\left\{ |\sigma f^{(\left\lceil 2^{\kappa-1}\right\rceil )}\sigma^{-1}(i)-i|:i\in\mathbb{Z}_{n}\right\} \right|.
\end{equation}
This contradicts the assertion of statement (i), thereby
establishing the backward claim. 
The exact same reasoning as above establishes (ii) $\iff$ (iii), for we have
\begin{equation}
        P_{f^{\left(\left\lceil 2^{\text{lg}(n-1)}\right\rceil \right)}} ({\bf x}) \not\equiv 0. 
\end{equation}
\end{proof}

Having assembled together the pieces required to prove our main result, we proceed to fit the pieces together to state and prove the \textit{Composition Lemma}.
\begin{lemma}[Composition Lemma]\label{lem:composition-lemma}
    %\lean{}
  %\leanok
  %\uses{defn:graceful-functional-graphs, defn:graceful-functional-graphs-set, prop:multivariate-quotient-remainder, defn:polynomial-grace-definition, defn:polynomial-congruence, prop:polynomial-grace-certificate, prop:composition-lemma-ineq}
 For all contractive  $f\in\mathbb{Z}_{n}^{\mathbb{Z}_{n}}$, i.e., functions subject to the fixed point condition
$\left|f^{\left(n-1\right)}\left(\mathbb{Z}_{n}\right)\right|=1$,
we have
\begin{equation}\label{eq:composition-lemma}
\max_{\sigma\in\text{S}_{n}}\big|\{|\sigma f^{(2)}\sigma^{-1}(i)-i|:i\in\mathbb{Z}_{n}\}\big|\le\max_{\sigma\in\text{S}_{n}}\big|\{|\sigma f\sigma^{-1}(i)-i|:i\in\mathbb{Z}_{n}\}\big|.
\end{equation}
\end{lemma}
\begin{proof}
Owing to Proposition \ref{prop:composition-lemma-ineq}, we prove the statement by establishing 
\[
    P_{f^{(2)}} ({\bf x}) \not\equiv 0  \implies  P_{f} ({\bf x}) \not\equiv 0.
\]

Recall that $\mathbb{Z}_{n}$ denote the smallest set of consecutive
$n$ non-negative integers. Let $\mathcal{T}_{n}\subset\mathbb{Z}_{n}^{\mathbb{Z}_{n}}$
denotes the semigroup of functions whose $n$ $(x,y)$\textendash coordinate
graph points lie above the line $y=x$ with the exception of an intercept
at $x=n-1$. In other words 
\begin{equation}
\mathcal{T}_{n}:=\big\{ h\in\mathbb{Z}_{n}^{\mathbb{Z}_{n}}:h(i)>i,\:\forall\,i\in\mathbb{Z}_{n-1}\text{ and }h(n-1)=n-1\big\}.
\end{equation}
 We prove the C.L. via the polynomial method.

\subsection*{Part I. Linear-algebra setup (matrices, kernels, and pseudoinverse)}

Associate with an arbitrary $h\in\mathbb{Z}_{n}^{\mathbb{Z}_{n}}$
the adjacency matrix $A_{h}\in\{0,1\}^{n\times n}$ of $G_{h}$ (the
functional directed graph of $h$): 
\begin{equation}
A_{h}[i,j]=\begin{cases}
\begin{array}{cc}
1 & \text{ if }j=h(i)\\
0 & \text{otherwise}
\end{array}, & \forall\:(i,j)\in\mathbb{Z}_{n}\times\mathbb{Z}_{n}.\end{cases}
\end{equation}
The signed incidence matrix of $G_{h}$ is $(A_{\text{id}}-A_{h})\in\{-1,0,1\}^{n\times n}$.
Observe that generally for all $h\in\mathbb{Z}_{n}^{\mathbb{Z}_{n}}$,
the rank of $(A_{\text{id}}-A_{h})$ is equal to $n$ minus the number
of connected component of $G_{h}$. In fact each basis vector for
the Nullspace of $(A_{\text{id}}-A_{h})$ is prescribed by the vertex
indicator vector of a corresponding connected component. Observe that
adjacency matrices of functional graphs of members of $\mathcal{T}_{n}$
are all upper-triangular matrices and their incidence matrices are
in Row Echelon Form. Furthermore, incidence matrices of functional
directed graphs of members of $\mathcal{T}_{n}$ all have the same
Reduced Row Echelon Form matrix. Namely 
\begin{equation}
\text{RREF}(A_{\text{id}}-A_{h})=A_{\text{id}}-A_{h^{(n-1)}},\quad\forall\,h\in\mathcal{T}_{n}.
\end{equation}
Observe that for all $h\in\mathcal{T}_{n}$, the function $h^{(n-1)}$
is the same identically constant function. Namely the function which
maps every member of $\mathbb{Z}_{n}$ to $n-1$. For all $h\in\mathcal{T}_{n}$,
the left and right kernel of $A_{\text{id}}-A_{h}$ are respectively
$A_{\text{id}}[n-1,:]$ and $1_{n\times1}$. Recall that a solution
exist to the equation in the unknown vector $\mathbf{z}$ 
\begin{equation}
B\cdot\mathbf{z}=\mathbf{c}
\end{equation}
where $\mathbf{c}\in\text{Column Space}(B)$. We adopt the following
notation convention for expressing solutions in the unknown vector
$\mathbf{z}$
\begin{equation}
\mathbf{z}\in B^{+}\cdot\mathbf{c}+\text{Null Space}(B),
\end{equation}
where $B^{+}=(B^{\top}\cdot B)^{-1}B^{\top}$ .

\pagebreak{}

\subsection*{Part II. Polynomial $f$-certificate}

Injectively assign to members of $\mathcal{T}_{n}$, a polynomial
which expresses a function of $n$ input vectors $\mathbf{x},\,\mathbf{b}_{0},\,\ldots,\,\mathbf{b}_{n-2}$
as follows
\[
P_{f}(\mathbf{x},\,y\mathbf{b}_{0},\,\ldots,\,y\mathbf{b}_{n-2})=
\]
\begin{equation}
\prod_{0\le u<v<n}\big(\sum_{w\in\mathbb{Z}_{n-1}}(B^{+}\cdot\mathbf{c}_{f,u,v})_{w}\,y\mathbf{b}_{w}^{\top}\cdot(A_{\text{id}}-A_{f})\cdot\mathbf{x}\big)\prod_{t\in\mathbb{Z}_{2}}\bigg(\big(A_{\text{id}}[v,:]\cdot(A_{\text{id}}-A_{f})+(-1)^{t}A_{\text{id}}[u,:]\cdot(A_{\text{id}}-A_{f})\big)\cdot\mathbf{x}\bigg),
\end{equation}
where $(B^{+}\cdot\mathbf{c}_{f,u,v})_{w}$ denotes the entry $w$
of the vector $(B^{+}\cdot\mathbf{c}_{f,u,v})$. Crucially vectors
column vectors $\big\{ B[:,0]=\mathbf{b}_{0},\ldots,B[:,n-2]=\mathbf{b}_{n-2}\big\}$
forms a basis for the $(n-1)$-dimensional subspace: $(\text{Span of }A_{\text{id}}[:,n-1])^{\perp}\subset\mathbb{Q}^{n\times1}$ 

\noindent For each one of the ${n \choose 2}$ vertex pair $(u,v)\in\mathbb{Z}_{n}\times\mathbb{Z}_{n}$
where $0\le u<v<n$, the vector $\mathbf{c}_{f,u,v}\in\{-1,0,1\}^{n\times1}$
denotes the unique vector subject to
\begin{equation}
\mathbf{c}_{f,u,v}^{\top}\cdot(A_{\text{id}}-A_{f})\cdot\mathbf{x}=(x_{v}-x_{u}).
\end{equation}
Let $H$ denote the projection matrix
\begin{equation}
H=(A_{\text{id}}-A_{\text{id}}[:,n-1]\cdot A_{\text{id}}[n-1,:])
\end{equation}
 We call the expression of $P_{f}$ above an expansion of the $f$-certificate
with respect to an arbitrarily chosen basis vectors $\{\mathbf{b}_{0},\,\ldots,\,\mathbf{b}_{n-2}\}$
for $(\text{Span of }A_{\text{id}}[:,n-1])^{\perp}$. Recall that
\begin{equation}
\bigg(\text{Vandermonde}\left(\begin{array}{c}
y\,x_{0}\\
\vdots\\
y\,x_{u}\\
\vdots\\
y\,x_{n-1}
\end{array}\right)\bigg)[i,j]=(y\,x_{i})^{j},\quad\forall\,0\le i,j<n\implies\det\bigg(\text{Vandermonde}\left(\begin{array}{c}
y\,x_{0}\\
\vdots\\
y\,x_{u}\\
\vdots\\
y\,x_{n-1}
\end{array}\right)\bigg)=y^{{n \choose 2}}\prod_{0\le i<j<n}(x_{j}-x_{i}),
\end{equation}
and 
\begin{equation}
\det\bigg(\text{Vandermode}\left(\begin{array}{c}
(x_{0}-x_{f(0)})^{2}\\
\vdots\\
(x_{u}-x_{f(u)})^{2}\\
\vdots\\
(x_{n-1}-x_{f(n-1)})^{2}
\end{array}\right)\bigg)=\prod_{0\le u<v<n}\big((x_{v}-x_{f(v)})^{2}-(x_{u}-x_{f(u)})^{2}\big).
\end{equation}
By construction, for any choices of basis vectors $\{\mathbf{b}_{0},\,\ldots,\,\mathbf{b}_{n-2}\}$
for $(\text{Span of }A_{\text{id}}[:,n-1])^{\perp}$ the following
$f$-certificate equality holds
\begin{equation}
P_{f}(\mathbf{x},y\mathbf{b}_{0},\ldots,y\mathbf{b}_{n-2})=\det\bigg(\text{Vandermonde}\left(\begin{array}{c}
y\,x_{0}\\
\vdots\\
y\,x_{u}\\
\vdots\\
y\,x_{n-1}
\end{array}\right)\bigg)\,\det\bigg(\text{Vandermonde}\left(\begin{array}{c}
(x_{0}-x_{f(0)})^{2}\\
\vdots\\
(x_{u}-x_{f(u)})^{2}\\
\vdots\\
(x_{n-1}-x_{f(n-1)})^{2}
\end{array}\right)\bigg).
\end{equation}


\subsection*{Part III. Change of variables and the relation between the $f$-certificate
and the $f^{(2)}$-certificate }

Consider the linear transformation prescribed by simultaneous maps:
\begin{equation}
\mathbf{x}\mapsto(A_{\text{id}}+A_{f})\cdot\mathbf{x},\quad y\mathbf{b}_{u}^{\top}\mapsto y\mathbf{b}_{u}^{\top}\cdot(A_{\text{id}}+A_{f})^{-1},\,\forall\,u\in\mathbb{Z}_{n-1},
\end{equation}
rewritten simply using $M_{f}:=(A_{\text{id}}+A_{f})$ as
\begin{equation}
\mathbf{x}\mapsto M_{f}\cdot\mathbf{x},\quad y\mathbf{b}_{u}^{\top}\mapsto y\mathbf{b}_{u}^{\top}\cdot M_{f}^{-1},\,\forall\,u\in\mathbb{Z}_{n-1}.
\end{equation}
The simultaneous maps are more conveniently summarized via the single
action of an invertible $n^{2}\times n^{2}$ matrix as follows: 
\begin{equation}
\left(\begin{array}{c}
\mathbf{x}\\
y\,\mathbf{b}_{0}\\
\vdots\\
y\,\mathbf{b}_{i}\\
\vdots\\
y\,\mathbf{b}_{n-2}
\end{array}\right)\mapsto\bigg(M_{f}\oplus\big(I_{n-1}\otimes(M_{f}^{\top})^{-1}\big)\bigg)\cdot\left(\begin{array}{c}
\mathbf{x}\\
y\,\mathbf{b}_{0}\\
\vdots\\
y\,\mathbf{b}_{i}\\
\vdots\\
y\,\mathbf{b}_{n-2}
\end{array}\right).
\end{equation}


\subsection*{Part III.a. Rewrites of the vertex and edge factors of the $f$-certificate}

\noindent Observe that for each triplet $u,v,w$ where $0\le u<v<n$
and $w\in\mathbb{Z}_{n-1}$, the following bilinear form equality
holds in the expression of the vertex Vandermonde factor 
\begin{equation}
(B^{+}\cdot\mathbf{c}_{f,u,v})_{w}\cdot y\mathbf{b}_{w}^{\top}\cdot(A_{\text{id}}-A_{f})\cdot\mathbf{x}=\left(\begin{array}{c}
\mathbf{x}\\
y\,\mathbf{b}_{0}\\
\vdots\\
y\,\mathbf{b}_{i}\\
\vdots\\
y\,\mathbf{b}_{n-2}
\end{array}\right)^{\top}\cdot\big(A_{\text{id}}[:,w]\cdot A_{\text{id}}[0,:]\big)\otimes\big((B^{+}\cdot\mathbf{c}_{f,u,v})_{w}\,(A_{\text{id}}-A_{f})\big)\cdot\left(\begin{array}{c}
\mathbf{x}\\
y\,\mathbf{b}_{0}\\
\vdots\\
y\,\mathbf{b}_{i}\\
\vdots\\
y\,\mathbf{b}_{n-2}
\end{array}\right),
\end{equation}
Similarly for all $u\in\mathbb{Z}_{n}$, the following equality holds
in the expression of the edge Vandermonde factor 
\begin{equation}
A_{\text{id}}[u,:]\cdot(A_{\text{id}}-A_{f})\cdot\mathbf{x}=\big(A_{\text{id}}[:,0]\otimes A_{\text{id}}[:,u]\big)^{\top}\cdot\big((A_{\text{id}}[:,0]\cdot A_{\text{id}}[0,:])\otimes(A_{\text{id}}-A_{f})\big)\cdot\left(\begin{array}{c}
\mathbf{x}\\
y\,\mathbf{b}_{0}\\
\vdots\\
y\,\mathbf{b}_{i}\\
\vdots\\
y\,\mathbf{b}_{n-2}
\end{array}\right).
\end{equation}
The two families of equalities described immediately above articulate
the fact that the $f$-certificate may be viewed as function the $n^{2}\times1$
vector 
\[
\left(\begin{array}{c}
\mathbf{x}\\
y\,\mathbf{b}_{0}\\
\vdots\\
y\,\mathbf{b}_{i}\\
\vdots\\
y\,\mathbf{b}_{n-2}
\end{array}\right)
\]
For further notational convenience we subsequently write $M$ to denote
$M_{f}$. We now show that applying the said linear transformation
\begin{equation}
\left(\begin{array}{c}
\mathbf{x}\\
y\,\mathbf{b}_{0}\\
\vdots\\
y\,\mathbf{b}_{i}\\
\vdots\\
y\,\mathbf{b}_{n-2}
\end{array}\right)\mapsto\bigg(M_{f}\oplus\big(I_{n-1}\otimes(M_{f}^{\top})^{-1}\big)\bigg)\cdot\left(\begin{array}{c}
\mathbf{x}\\
y\,\mathbf{b}_{0}\\
\vdots\\
y\,\mathbf{b}_{i}\\
\vdots\\
y\,\mathbf{b}_{n-2}
\end{array}\right),
\end{equation}
maps the expansion of a $f$-certificate with respect to the basis
$\{\mathbf{b}_{0},\ldots,\mathbf{b}_{n-2}\}$ of $(\text{Span of }A_{\text{id}}[:,n-1])^{\perp}$
denoted 
$$P_{f}(\mathbf{x},\,y\,\mathbf{b}_{0},\,\ldots,\,y\,\mathbf{b}_{n-2})$$
to the expansion of a $f^{(2)}$-certificate with respect to the basis
\begin{equation}
\big\{\mathbf{b}_{0}^{\prime}=H(M^{\top})^{-1}\mathbf{b}_{0},\ldots,\mathbf{b}_{i}^{\prime}=H(M^{\top})^{-1}\mathbf{b}_{i},\ldots,\mathbf{b}_{n-2}^{\prime}=H(M^{\top})^{-1}\mathbf{b}_{n-2}\big\},
\end{equation}
of $(\text{Span of }A_{\text{id}}[:,n-1])^{\perp}$, denoted $P_{f^{(2)}}(\mathbf{x},y\mathbf{b}_{0}^{\prime},\ldots,y\mathbf{b}_{n-2}^{\prime})$.
Thus establishing that
\begin{equation}
P_{f^{(2)}}(\mathbf{x},y\mathbf{b}_{0}^{\prime},\ldots,y\mathbf{b}_{n-2}^{\prime})=\det\bigg(\text{Vandermonde}\left(\begin{array}{c}
y\,x_{0}\\
\vdots\\
y\,x_{u}\\
\vdots\\
y\,x_{n-1}
\end{array}\right)\bigg)\,\det\bigg(\text{Vandermonde}\left(\begin{array}{c}
(x_{0}-x_{f^{(2)}(0)})^{2}\\
\vdots\\
(x_{u}-x_{f^{(2)}(u)})^{2}\\
\vdots\\
(x_{n-1}-x_{f^{(2)}(n-1)})^{2}
\end{array}\right)\bigg).
\end{equation}


\subsection*{Part III.b. Proof of Expansion of $f^{(2)}$ certificate in the transformed
basis.}

\noindent Recall that an expansion of the $f^{(2)}$-certificate with
respect to the basis $\big\{\mathbf{b}_{0}^{\prime},\ldots,\mathbf{b}_{n-2}^{\prime}\big\}$
forms a basis of $(\text{Span of }A_{\text{id}}[:,n-1])^{\perp}$
is given by
\[
P_{f^{(2)}}(\mathbf{x},y\mathbf{b}_{0}^{\prime},\ldots,y\mathbf{b}_{n-2}^{\prime})=\prod_{0\le u<v<n}\bigg(\sum_{w\in\mathbb{Z}_{n-1}}\big((B^{\prime})^{+}\cdot\mathbf{c}_{f^{(2)},u,v}\big)_{w}\,y(\mathbf{b}_{w}^{\prime})^{\top}\cdot(A_{\text{id}}-A_{f^{(2)}})\cdot\mathbf{x}\bigg)\times
\]
\begin{equation}
\prod_{0\le u<v<n}\prod_{t\in\mathbb{Z}_{2}}\bigg(\big(A_{\text{id}}[v,:]\cdot(A_{\text{id}}-A_{f^{(2)}})+(-1)^{t}A_{\text{id}}[u,:]\cdot(A_{\text{id}}-A_{f^{(2)}})\big)\cdot\mathbf{x}\bigg).
\end{equation}
where $B^{\prime}=H(M^{\top})^{-1}B$ 
\[
=\prod_{0\le u<v<n}\bigg(\sum_{w\in\mathbb{Z}_{n-1}}\big((B^{\prime})^{+}\cdot\mathbf{c}_{f^{(2)},u,v}\big)_{w}\,y\mathbf{b}_{w}^{\top}\cdot M^{-1}\cdot(A_{\text{id}}-A_{f})\cdot M\cdot\mathbf{x}\bigg)\times
\]
\begin{equation}
\prod_{0\le u<v<n}\prod_{t\in\mathbb{Z}_{2}}\bigg(\big(A_{\text{id}}[v,:]\cdot(A_{\text{id}}-A_{f})+(-1)^{t}A_{\text{id}}[u,:]\cdot(A_{\text{id}}-A_{f})\big)\cdot M\cdot\mathbf{x}\bigg).
\end{equation}
Recall that $\mathbf{c}_{f,u,v}$ is the unique vector in $(\text{Span of }A_{\text{id}}[:,n-1])^{\perp}$
subject to
\begin{equation}
\mathbf{c}_{f,u,v}^{\top}(A_{\text{id}}-A_{f})\mathbf{x}=(x_{v}-x_{u}),
\end{equation}
Observe that 
\begin{equation}
\mathbf{c}_{f,u,v}^{\top}M^{-1}(A_{\text{id}}-A_{f^{(2)}})\mathbf{x}=\mathbf{c}_{f,u,v}^{\top}(M^{\top})^{-1}(A_{\text{id}}-A_{f})M\mathbf{x}=(x_{v}-x_{u}),
\end{equation}
However $(M^{\top})^{-1}\mathbf{c}_{f,u,v}$ does not necessarily
lie in $(\text{Span of }A_{\text{id}}[:,n-1])^{\perp}$. Since $(A_{\text{id}}[:,n-1])^{\top}$
lies in the left kernel of $(A_{\text{id}}-A_{f^{(2)}})$ adding a
multiple of $A_{\text{id}}[:,n-1]$ to $\mathbf{c}_{f,u,v}^{\top}M^{-1}$
does not affect the equality immediately above. Thus for all $0\le u<v<n$
we have
\begin{equation}
\mathbf{c}_{f^{(2)},u,v}=H(M^{\top})^{-1}\mathbf{c}_{f,u,v}.
\end{equation}
Now fix a basis matrix $B=[\mathbf{b}_{0}\,\cdots\,\mathbf{b}_{n-2}]$
of the subspace $(\text{Span of }A_{\text{id}}[:,n-1])^{\perp}$ and
define the transformed basis matrix to be
\begin{equation}
B^{\prime}:=H\,M^{-T}B=[\mathbf{b}_{0}^{\prime}\,\cdots\,\mathbf{b}_{n-2}^{\prime}]\quad\text{ with }\quad\mathbf{b}^{\prime}{}_{w}:=H(M^{\top})^{-1}\mathbf{b}_{w}\in(\text{Span of }A_{\text{id}}[:,n-1])^{\perp}.
\end{equation}
\begin{equation}
\implies(B^{\prime})^{+}\mathbf{c}_{f^{(2)},u,v}=B^{+}\mathbf{c}_{f,u,v}.
\end{equation}
\[
P_{f^{(2)}}\big(\mathbf{x},y\mathbf{b}_{0}^{\prime},\ldots,y\mathbf{b}_{n-2}^{\prime}\big)=\prod_{0\le u<v<n}\left(\sum_{w\in\mathbb{Z}_{n-1}}(B^{+}\cdot\mathbf{c}_{f,u,v})_{w}\,y\mathbf{b}_{w}^{\top}\cdot M^{-1}\cdot(A_{\text{id}}-A_{f})\cdot M\cdot\mathbf{x}\right)\times
\]
\begin{equation}
\prod_{0\le u<v<n}\prod_{t\in\mathbb{Z}_{2}}\bigg(\big(A_{\text{id}}[v,:]\cdot(A_{\text{id}}-A_{f})+(-1)^{t}A_{\text{id}}[u,:]\cdot(A_{\text{id}}-A_{f})\big)\cdot M\cdot\mathbf{x}\bigg).
\end{equation}
We see that the right hand side of the equality immediately above
is devised from the expansion of the $f$-certificate with respect
to the chosen basis $\{\mathbf{b}_{0},\ldots,\mathbf{b}_{n-2}\}$
by carrying the linear transformation
\begin{equation}
\left(\begin{array}{c}
\mathbf{x}\\
y\,\mathbf{b}_{0}\\
\vdots\\
y\,\mathbf{b}_{i}\\
\vdots\\
y\,\mathbf{b}_{n-2}
\end{array}\right)\mapsto\bigg(M\oplus\big(I_{n-1}\otimes(M^{\top})^{-1}\big)\bigg)\cdot\left(\begin{array}{c}
\mathbf{x}\\
y\,\mathbf{b}_{0}\\
\vdots\\
y\,\mathbf{b}_{i}\\
\vdots\\
y\,\mathbf{b}_{n-2}
\end{array}\right).
\end{equation}


\subsection*{Part III.c. Remark on scalars vs basis change. }

\noindent We emphasize to the reader that the said linear transformation
does not affect scalars
\begin{equation}
\{(B^{+}\cdot\mathbf{c}_{f,u,v})_{w}\,:\,w\in\mathbb{Z}_{n-1}\}.
\end{equation}
Thus the map prescribed by
\begin{equation}
\left(\begin{array}{c}
\mathbf{x}\\
y\,\mathbf{b}_{0}\\
\vdots\\
y\,\mathbf{b}_{i}\\
\vdots\\
y\,\mathbf{b}_{n-2}
\end{array}\right)\mapsto\bigg(I_{n}\oplus\big(I_{n-1}\otimes(M^{\top})^{-1}\big)\bigg)\cdot\left(\begin{array}{c}
\mathbf{x}\\
y\,\mathbf{b}_{0}\\
\vdots\\
y\,\mathbf{b}_{i}\\
\vdots\\
y\,\mathbf{b}_{n-2}
\end{array}\right)
\end{equation}
in this particular setting does not equate to a base change in the
expansion of the $f$-certificate with respect to the basis $\{\mathbf{b}_{0},\ldots,\mathbf{b}_{n-2}\}$.
For we know that a change of basis must also affect scalars
\begin{equation}
\{(B^{+}\cdot\mathbf{c}_{f,u,v})_{w}\,:\,w\in\mathbb{Z}_{n-1}\},
\end{equation}
in that they change to coefficients with respect to the new chosen
basis vectors.

\subsection*{Part IV. Canonical representatives modulo the ``grid'' ideal forcing
$x_{i}\in\mathbb{Z}_{n}$ }

\noindent Canonical representatives (i.e. remainders) of $P_{f}(\mathbf{x},y\mathbf{b}_{0},\ldots,y\mathbf{b}_{n-2})$
and $P_{f^{(2)}}(\mathbf{x},y\mathbf{b}_{0}^{\prime},\ldots,y\mathbf{b}_{n-2}^{\prime})$
modulo the polynomial ideal generated by $\{\underset{j\in\mathbb{Z}_{n}}{\prod}(x_{i}-j):i\in\mathbb{Z}_{n}\}$
are respectively 
\begin{equation}
\overline{P_{f}(\mathbf{x},y\mathbf{b}_{0},\ldots,y\mathbf{b}_{n-2})}=y^{{n \choose 2}}\prod_{v\in\mathbb{Z}_{n}}\big((v!)^{2}\frac{(n-1+v)!}{(2v)!}\big)\sum_{\begin{array}{c}
\sigma\in\text{S}_{n}\\
|\sigma f-\sigma|\in\text{S}_{n}
\end{array}}\text{sgn}(\sigma\circ|\sigma f-\sigma|)\,\prod_{i\in\mathbb{Z}_{n}}\big(\prod_{j_{i}\in\mathbb{Z}_{n}\backslash\{\sigma(i)\}}(\frac{x_{i}-j_{i}}{\sigma(i)-j_{i}})\big),
\end{equation}
and 
\[
\overline{P_{f^{(2)}}(\mathbf{x},y\mathbf{b}_{0}^{\prime},\ldots,y\mathbf{b}_{n-2}^{\prime})}=
\]
\begin{equation}
y^{{n \choose 2}}\prod_{v\in\mathbb{Z}_{n}}\big((v!)^{2}\frac{(n-1+v)!}{(2v)!}\big)\sum_{\begin{array}{c}
\sigma\in\text{S}_{n}\\
|\sigma f^{(2)}-\sigma|\in\text{S}_{n}
\end{array}}\text{sgn}(\sigma\circ|\sigma f^{(2)}-\sigma|)\,\prod_{i\in\mathbb{Z}_{n}}\big(\prod_{j_{i}\in\mathbb{Z}_{n}\backslash\{\sigma(i)\}}(\frac{x_{i}-j_{i}}{\sigma(i)-j_{i}})\big).
\end{equation}
Observe from the expression immediately above of the canonical representative
that non-vanishing evaluations over the lattice $\mathbb{Z}_{n}^{\mathbb{Z}_{n}}$
of $P_{f}$ and $P_{f^{(2)}}$ are all equal up to sign and congruent
to zero modulo $n$.

\subsection*{Part V. Finite field variant of $f$-certificate over $\mathbb{F}_{p}$
(to avoid the mod-$n$ degeneracy)}

\noindent In the subsequent part of the argument we bypass investigations
of ring homomorphisms (induced by taking the quotient of the ring
$\mathbb{Q}[x_{0},\ldots,x_{n-1}]$ modulo the ideal generated by
members of $\big\{\underset{j\in\mathbb{Z}_{n}}{\prod}(x_{i}-j):i\in\mathbb{Z}_{n}\big\}$)
by switching the ground field from $\mathbb{Q}$ (i.e. rational numbers)
to the Galois field of order $p=\left\lceil 2n-1\right\rceil _{\mathbb{P}}$
(i.e. the smallest prime number greater than $2n-2$). We point out
to the reader that any prime $p>2n-2$ would work just as well in
the our argument. We proceeds with the following variant of the $f$-certificate
given by
\begin{equation}
\mathcal{P}_{f}(\mathbf{x})=\left(\prod_{0\le u<v<n}(x_{v}-x_{u})\right)\left(\prod_{0\le u<v<n}\Big((x_{f(v)}-x_{v})^{2}-(x_{f(u)}-x_{u})^{2}\Big)\right)\left(\prod_{i\in\mathbb{Z}_{n}}\bigg(\prod_{j_{i}\in\mathbb{Z}_{p}\backslash\mathbb{Z}_{n}}(x_{i}-j_{i})\bigg)\right).
\end{equation}
and the congruence class of interest is
\begin{equation}
\mathcal{P}_{f}\big(\mathbf{x})\mod\big\{\prod_{j\in\mathbb{Z}_{p}}(x_{i}-j):i\in\mathbb{Z}_{n}\big\}.
\end{equation}
Note that every non vanishing evaluation point occurs at a lattice
point of $\mathbb{Z}_{n}^{\mathbb{Z}_{n}}$ having exactly one zero
entry. By construction the congruence class immediately above vanishes
identically if and only if for all $r\in\mathbb{Z}_{n}$ the congruence
class 
\[
\bigg(\prod_{\begin{array}{c}
0\le u<v<n\\
r\notin\{u,v\}
\end{array}}(x_{v}-x_{u})\bigg)\bigg(\prod_{w\in\mathbb{Z}_{n}\backslash\{r\}}x_{w}\bigg)\left(\prod_{i\in\mathbb{Z}_{n}\backslash\{r\}}\bigg(\prod_{j_{i}\in\mathbb{Z}_{p}\backslash\mathbb{Z}_{n}}(x_{i}-j_{i})\bigg)\right)\times
\]
\begin{equation}
\prod_{0\le u<v<n}\bigg(\prod_{t\in\mathbb{Z}_{2}}\big(A_{\text{id}}[v,:](A_{\text{id}}-A_{f})+(-1)^{t}A_{\text{id}}[u,:](A_{\text{id}}-A_{f})\big)\left(\begin{array}{c}
x_{0}\\
\vdots\\
x_{r-1}\\
0\\
x_{r+1}\\
\vdots\\
x_{n-1}
\end{array}\right)\bigg)\mod\big\{\prod_{j\in\mathbb{Z}_{p}}(x_{i}-j):i\in\mathbb{Z}_{n}\backslash\{r\}\big\}
\end{equation}
vanishes identically. We make sense of the linear transformation by
defining the expansion of a slight variant of the $f$-certificate
that we call the $(f,r)$-certificate relative to an arbitrarily chosen
basis $\{\mathbf{b}_{0},\,\ldots,\,\mathbf{b}_{n-2}\}$ of $(\text{Span of }A_{\text{id}}[:,n-1])^{\perp}$
and for some arbitrary $r\in\mathbb{Z}_{n}$ as follows:
\[
\mathcal{P}_{f,r}\big(\left(\begin{array}{c}
x_{0}\\
\vdots\\
x_{r-1}\\
0\\
x_{r+1}\\
\vdots\\
x_{n-1}
\end{array}\right),y\mathbf{b}_{0},\ldots,y\mathbf{b}_{n-2}\big)=\prod_{0\le u<v<n}\big(\sum_{w\in\mathbb{Z}_{n-1}}(B^{+}\mathbf{c}_{f,u,v})_{w}\,y\mathbf{b}_{w}^{\top}(A_{\text{id}}-A_{f})\left(\begin{array}{c}
x_{0}\\
\vdots\\
x_{r-1}\\
0\\
x_{r+1}\\
\vdots\\
x_{n-1}
\end{array}\right)\big)\times
\]
\[
\prod_{i<r}\bigg(\prod_{j\in\mathbb{Z}_{p}\backslash\mathbb{Z}_{n}}\big(-\sum_{w\in\mathbb{Z}_{n-1}}(B^{+}\mathbf{c}_{f,i,r})_{w}y\mathbf{b}_{w}^{\top}(A_{\text{id}}-A_{f})\left(\begin{array}{c}
x_{0}\\
\vdots\\
x_{r-1}\\
0\\
x_{r+1}\\
\vdots\\
x_{n-1}
\end{array}\right)-yj\big)\bigg)\prod_{i>r}\bigg(\prod_{j\in\mathbb{Z}_{p}\backslash\mathbb{Z}_{n}}\big(\sum_{w\in\mathbb{Z}_{n-1}}(B^{+}\mathbf{c}_{f,r,i})_{w}y\mathbf{b}_{w}^{\top}(A_{\text{id}}-A_{f})\left(\begin{array}{c}
x_{0}\\
\vdots\\
x_{r-1}\\
0\\
x_{r+1}\\
\vdots\\
x_{n-1}
\end{array}\right)-yj\big)\bigg)\times
\]
\begin{equation}
\prod_{0\le u<v<n}\bigg(\prod_{t\in\mathbb{Z}_{2}}\big(A_{\text{id}}[v,:](A_{\text{id}}-A_{f})+(-1)^{t}A_{\text{id}}[u,:](A_{\text{id}}-A_{f})\big)\left(\begin{array}{c}
x_{0}\\
\vdots\\
x_{r-1}\\
0\\
x_{r+1}\\
\vdots\\
x_{n-1}
\end{array}\right)\bigg).
\end{equation}
Just as in the previous setting, for all basis vectors $\{\mathbf{b}_{0},\,\ldots,\,\mathbf{b}_{n-2}\}$
of $(\text{Span of }A_{\text{id}}[:,n-1])^{\perp}$ we have 
\[
\mathcal{P}_{f,r}\big(\left(\begin{array}{c}
x_{0}\\
\vdots\\
x_{r-1}\\
0\\
x_{r+1}\\
\vdots\\
x_{n-1}
\end{array}\right),y\mathbf{b}_{0},\ldots,y\mathbf{b}_{n-2}\big)=y^{{n \choose 2}+(n-1)(p-n)}\bigg(\prod_{\begin{array}{c}
0\le u<v<n\\
r\notin\{u,v\}
\end{array}}(x_{v}-x_{u})\bigg)(\prod_{w\in\mathbb{Z}_{n}\backslash\{r\}}x_{w})\bigg(\prod_{i\in\mathbb{Z}_{n}\backslash\{r\}}\big(\prod_{j_{i}\in\mathbb{Z}_{p}\backslash\mathbb{Z}_{n}}(x_{i}-j_{i})\big)\bigg)\times
\]
\begin{equation}
\prod_{0\le u<v<n}\bigg(\prod_{t\in\mathbb{Z}_{2}}\big(A_{\text{id}}[v,:](A_{\text{id}}-A_{f})+(-1)^{t}A_{\text{id}}[u,:](A_{\text{id}}-A_{f})\big)\left(\begin{array}{c}
x_{0}\\
\vdots\\
x_{r-1}\\
0\\
x_{r+1}\\
\vdots\\
x_{n-1}
\end{array}\right)\bigg).
\end{equation}


\subsection*{Part V.a. Explicit expression of $(f,r)$-certificates in terms of
the $f$-certificate. }

\noindent By construction
\[
\mathcal{P}_{f,r}\big(\left(\begin{array}{c}
x_{0}\\
\vdots\\
x_{r-1}\\
0\\
x_{r+1}\\
\vdots\\
x_{n-1}
\end{array}\right),y\mathbf{b}_{0},\ldots,y\mathbf{b}_{n-2}\big)=
\]
\begin{equation}
\bigg(\prod_{\begin{array}{c}
i>r\\
j_{i}\in\mathbb{Z}_{p}\backslash\mathbb{Z}_{n}
\end{array}}(y\,x_{i}-y\,x_{r}-j_{i}y)\prod_{\begin{array}{c}
i<r\\
j_{i}\in\mathbb{Z}_{p}\backslash\mathbb{Z}_{n}
\end{array}}\big(-(y\,x_{r}-y\,x_{i})-j_{i}y\big)\,P_{f}(\mathbf{x},y\mathbf{b}_{0},\ldots,y\mathbf{b}_{n-2})\mod x_{r}\bigg)
\end{equation}
its canonical representative is
\begin{equation}
y^{{n \choose 2}+(n-1)(p-n)}\prod_{v\in\mathbb{Z}_{n}}\big((v!)^{2}\frac{(n-1+v)!}{(2v)!}\big)\underset{v\in\mathbb{Z}_{n}\backslash\{r\}}{\prod}\big(\underset{u\in\mathbb{Z}_{p}\backslash\mathbb{Z}_{n}}{\prod}(v-u)\big)\sum_{\begin{array}{c}
\sigma\in\text{S}_{n}\\
\sigma(r)=0\\
|\sigma f-\sigma|\in\text{S}_{n}
\end{array}}\text{sgn}(\sigma\circ|\sigma f-\sigma|)\prod_{i\in\mathbb{Z}_{n}\backslash\{r\}}\big(\prod_{j_{i}\in\mathbb{Z}_{p}\backslash\{\sigma(i)\}}(\frac{x_{i}-j_{i}}{\sigma(i)-j_{i}})\big).
\end{equation}


\subsection*{Part V.b. Compatibility of $(f,r)$-certificate with the same change
of variables. }

\noindent Similarly to the previous setting, we consider for all $r\in\mathbb{Z}_{n}$
the linear transformation prescribed by the map
\begin{equation}
\left(\begin{array}{c}
\left(\begin{array}{c}
x_{0}\\
\vdots\\
x_{r-1}\\
0\\
x_{r+1}\\
\vdots\\
x_{n-1}
\end{array}\right)\\
y\,\mathbf{b}_{0}\\
\vdots\\
y\,\mathbf{b}_{i}\\
\vdots\\
y\,\mathbf{b}_{n-2}
\end{array}\right)\mapsto\bigg(M\oplus\big(I_{n-1}\otimes(M^{\top})^{-1}\big)\bigg)\cdot\left(\begin{array}{c}
\left(\begin{array}{c}
x_{0}\\
\vdots\\
x_{r-1}\\
0\\
x_{r+1}\\
\vdots\\
x_{n-1}
\end{array}\right)\\
y\,\mathbf{b}_{0}\\
\vdots\\
y\,\mathbf{b}_{i}\\
\vdots\\
y\,\mathbf{b}_{n-2}
\end{array}\right).
\end{equation}
The latter linear transformation maps the $(f,r)$-certificate relative
to an arbitrarily chosen basis $\{\mathbf{b}_{0},\,\ldots,\,\mathbf{b}_{n-2}\}$
of $(\text{Span of }A_{\text{id}}[:,n-1])^{\perp}$
\[
\mathcal{P}_{f,r}\big(\left(\begin{array}{c}
x_{0}\\
\vdots\\
x_{r-1}\\
0\\
x_{r+1}\\
\vdots\\
x_{n-1}
\end{array}\right),y\mathbf{b}_{0},\ldots,y\mathbf{b}_{n-2}\big)
\]
 to the $(f^{(2)},r)$-certificate relative to the basis $\big\{\mathbf{b}_{0}^{\prime},\ldots,\mathbf{b}_{n-2}^{\prime}\big\}$
of $(\text{Span of }A_{\text{id}}[:,n-1])^{\perp}$
\[
\mathcal{P}_{f^{(2)},r}\big(\left(\begin{array}{c}
x_{0}\\
\vdots\\
x_{r-1}\\
0\\
x_{r+1}\\
\vdots\\
x_{n-1}
\end{array}\right),y\mathbf{b}_{0}^{\prime},\ldots,y\mathbf{b}_{n-2}^{\prime}\big)=\prod_{0\le u<v<n}\big(\sum_{w\in\mathbb{Z}_{n-1}}(B^{+}\mathbf{c}_{f,u,v})_{w}\,y\mathbf{b}_{w}^{\top}M^{-1}(A_{\text{id}}-A_{f})M\left(\begin{array}{c}
x_{0}\\
\vdots\\
x_{r-1}\\
0\\
x_{r+1}\\
\vdots\\
x_{n-1}
\end{array}\right)\big)\times
\]
\[
\prod_{i<r}\bigg(\prod_{j\in\mathbb{Z}_{p}\backslash\mathbb{Z}_{n}}\big(-\sum_{w\in\mathbb{Z}_{n-1}}(B^{+}\mathbf{c}_{f,i,r})_{w}\,y\mathbf{b}_{w}^{\top}M^{-1}(A_{\text{id}}-A_{f})M\left(\begin{array}{c}
x_{0}\\
\vdots\\
x_{r-1}\\
0\\
x_{r+1}\\
\vdots\\
x_{n-1}
\end{array}\right)-yj\big)\bigg)\times
\]
\[
\prod_{i>r}\bigg(\prod_{j\in\mathbb{Z}_{p}\backslash\mathbb{Z}_{n}}\big(\sum_{w\in\mathbb{Z}_{n-1}}(B^{+}\mathbf{c}_{f,r,i})_{w}\,y\mathbf{b}_{w}^{\top}M^{-1}(A_{\text{id}}-A_{f})M\left(\begin{array}{c}
x_{0}\\
\vdots\\
x_{r-1}\\
0\\
x_{r+1}\\
\vdots\\
x_{n-1}
\end{array}\right)-yj\big)\bigg)\times
\]
\begin{equation}
\prod_{0\le u<v<n}\bigg(\prod_{t\in\mathbb{Z}_{2}}\big(A_{\text{id}}[v,:](A_{\text{id}}-A_{f})+(-1)^{t}A_{\text{id}}[u,:](A_{\text{id}}-A_{f})\big)M\left(\begin{array}{c}
x_{0}\\
\vdots\\
x_{r-1}\\
0\\
x_{r+1}\\
\vdots\\
x_{n-1}
\end{array}\right)\bigg).
\end{equation}


\subsection*{Part VI. Final implication and conclusion (invertible change preserves
identically-zero polynomials). }

\noindent The latter $(f,r)$-certificate and $(f^{(2)},r)$-certificate
favorably compare to the previous $f$-certificate and $f^{(2)}$-certificate
in the sense that their non-vanishing evaluations over $(\nicefrac{\mathbb{Z}}{p\mathbb{Z}})^{\mathbb{Z}_{n}}$
are not congruent to zero modulo $p$. We conclude the proof by showing
that the vanishing identically of the $(f,r)$-certificate over $\nicefrac{\mathbb{Z}}{p\mathbb{Z}}$
for all $r\in\mathbb{Z}_{n}$ implies the vanishing identically of
the $(f^{(2)},r)$-certificate over $\nicefrac{\mathbb{Z}}{p\mathbb{Z}}$
for all $r\in\mathbb{Z}_{n}$. The transformation prescribed by the
action 
\begin{equation}
\left(\begin{array}{c}
\left(\begin{array}{c}
x_{0}\\
\vdots\\
x_{r-1}\\
0\\
x_{r+1}\\
\vdots\\
x_{n-1}
\end{array}\right)\\
y\,\mathbf{b}_{0}\\
\vdots\\
y\,\mathbf{b}_{i}\\
\vdots\\
y\,\mathbf{b}_{n-2}
\end{array}\right)\mapsto\bigg(M\oplus\big(I_{n-1}\otimes(M^{\top})^{-1}\big)\bigg)\cdot\left(\begin{array}{c}
\left(\begin{array}{c}
x_{0}\\
\vdots\\
x_{r-1}\\
0\\
x_{r+1}\\
\vdots\\
x_{n-1}
\end{array}\right)\\
y\,\mathbf{b}_{0}\\
\vdots\\
y\,\mathbf{b}_{i}\\
\vdots\\
y\,\mathbf{b}_{n-2}
\end{array}\right).
\end{equation}
is alternatively carried out by an action of some larger induced invertible
matrix on the coefficient vector of the polynomial. By invertibility,
such an action necessarily maps each identically vanishing $(f,r)$-certificate
to an identically vanishing $(f^{(2)},r)$-certificate. Thereby resulting
in the contrapositive of the assertion of the Composition Lemma.
\begin{equation}
\bigg(\mathcal{P}_{f,r}\big(\left(\begin{array}{c}
x_{0}\\
\vdots\\
x_{r-1}\\
0\\
x_{r+1}\\
\vdots\\
x_{n-1}
\end{array}\right),\,y\mathbf{b}_{0},\,\ldots,\,y\mathbf{b}_{n-2}\big)\equiv0,\:\forall\,r\in\mathbb{Z}_{n}\bigg)\implies\bigg(\mathcal{P}_{f^{(2)},r}\big(\left(\begin{array}{c}
x_{0}\\
\vdots\\
x_{r-1}\\
0\\
x_{r+1}\\
\vdots\\
x_{n-1}
\end{array}\right),y\mathbf{b}_{0}^{\prime},\ldots,y\mathbf{b}_{n-2}^{\prime}\big)\equiv0,\:\forall\,r\in\mathbb{Z}_{n}\bigg).
\end{equation}
For each $r\in\mathbb{Z}_{n}$, congruence classes above are taken
modulo the ideal generated by $\big\{\underset{j\in\mathbb{Z}_{p}}{\prod}(x_{i}-j):i\in\mathbb{Z}_{n}\backslash\{r\}\big\}.$
Or equivalently we have
\begin{equation}
\bigg(\mathcal{P}_{f}(\mathbf{x})\equiv0\mod\big\{\prod_{j\in\mathbb{Z}_{p}}(x_{i}-j):i\in\mathbb{Z}_{n}\big\}\bigg)\implies\bigg(\mathcal{P}_{f^{(2)}}(\mathbf{x})\equiv0\mod\big\{\prod_{j\in\mathbb{Z}_{p}}(x_{i}-j):i\in\mathbb{Z}_{n}\big\}\bigg)
\end{equation}
Thus 
\begin{equation}
\big(\text{GrL}(G_{f})=\varnothing\implies\text{GrL}(G_{f^{(2)}})=\varnothing\big)\Longleftrightarrow\big(\text{GrL}(G_{f^{(2)}})\ne\varnothing\implies\text{GrL}(G_{f})\ne\varnothing\big).
\end{equation} 
\end{proof}

\begin{example} The figure below illustrates the local iteration described in the proof of Lemma \ref{lem:composition-lemma} with an example of a path on 5 vertices.
\ \\
\begin{center}   $G_f: \;$
\resizebox{!}{3em}{
\begin{tikzpicture}[>={Stealth[round]}, node distance=2cm and 1cm, every node/.style={circle, draw, minimum size=1cm}]
    % Define vertices
    \node (v0) at (0, 0) {4};
    \node (v1) at (3, 0) {3};
    \node (v2) at (6, 0) {2};
    \node (v3) at (9, 0) {1};
    \node (v4) at (12, 0) {0};
    
    % Draw edges
    \draw [->] (v1) to (v0); % Edge 1 -> 0
    \draw [->] (v2) to (v1); % Edge 2 -> 1
    \draw [->] (v3) to (v2); % Edge 3 -> 2
    \draw [->] (v4) to (v3); % Edge 4 -> 3
    \draw[->] (v0) edge[loop above] (v0); % Self-loop at 0
\end{tikzpicture}
}
\end{center}
\ \\
\begin{center}   
$G_g: \;$
\resizebox{!}{3em}{
\begin{tikzpicture}[>={Stealth[round]}, node distance=2cm and 1cm, every node/.style={circle, draw, minimum size=1cm}]
    % Define vertices
    \node (v0) at (0, 0) {4};
    \node (v1) at (3, 0) {3};
    \node (v2) at (6, 0) {2};
    \node (v3) at (9, 0) {1};
    \node (v4) at (12, 0) {0};
    
    % Draw edges
    \draw [->] (v1) to (v0); % Edge 1 -> 0
    \draw [->] (v2) to (v1); % Edge 2 -> 1
    \draw [->] (v3) to (v2); % Edge 3 -> 2
    \draw [->] (v4) edge[bend right] (v2); % Edge 4 -> 2
    \draw[->] (v0) edge[loop above] (v0); % Self-loop at 0
\end{tikzpicture}
}
\end{center}
\end{example}
